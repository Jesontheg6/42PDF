
% vim: set ts=4 sw=4 tw=80 noexpandtab:
%******************************************************************************%
%                                                                              %
%                   TicTacToe.tex                                              %
%                   Made by: 42 staff                                          %
%                                                                              %
%******************************************************************************%

\documentclass{42-en}

%******************************************************************************%
%                                                                              %
%                                   Prologue                                   %
%                                                                              %
%******************************************************************************%

\begin{document}

\title{Tic Tac Toe}
\subtitle{Command line games}

\member {Kai}{kai@42.us.org}

\summary
{
The art of building a program.
}

\maketitle

\tableofcontents

%Initialisation des headers d'exercices

\startexercices


%******************************************************************************%
%                                                                              %
%                            General Instructions                              %
%                                                                              %
%******************************************************************************%

\chapter{Gameboard}

\begin{itemize}

	\item How do we play tic tact toe? It starts with drawing a \# ... two sets of perpendicular lines. A grid.

	\item Decide for yourself how to design a game board. It could look something like this:
	\begin{42console}
	 x | o |
	-----------
	   | o | o
	-----------
	   |   | x 
	\end{42console}

	\item Be careful and plan ahead! As the game progresses, we want to be able to store and display values of o, x, or nil in each square.

	\item How will you save the data? It could work multiple ways. For example, you could use nine different variables...
	\begin{42ccode}
	square_0 = nil;
	square_1 = nil;
	square_2 = 'x';
	square_3 = nil;
	...
	\end{42ccode}

	\item Or an array...
	\begin{42ccode}
	board = Array.new
	board[2] = 'x'
	\end{42ccode}

	\item Or an array of arrays if you really want.
	\begin{42ccode}
	board = Array.new
	top_row = Array.new
	board[0] = top_row
	...
	\end{42ccode}

	\item Choose the data structure that will hold your game board, and add it to your program. Let the whole board start "empty" with no x's or o's. 

\end{itemize}

\chapter{Printing}

\begin{itemize}

	\item When you first start the tictac.rb, it should print a welcome message to the terminal, and the empty game board.

	\begin{42console}
	?> tictac.rb
	Welcome to Intergalactic Tic Tac Toe!
	   |   |   
	-----------
	   |   |   
	-----------
	   |   |   
	\end{42console}

	\item Later, you will want to print the same game board with something else in the squares besides empty spaces.

	\item Create a \texttt{method} in your program which takes the state of the board as parameter(s) and prints it out.

	\item Test it by changing the board state a few times within your program, and printing out each version.

	\begin{42ccode}
	def print_board_parameters(board)
		#your method definition here
	end

	# blank board
	print_board(board)

	board[2] = 'x'

	# after one move
	print_board(board)
	\end{42ccode}

\end{itemize}

\chapter{Making the First Move}

\begin{itemize}

	\item OK, that's cool. You can print out different versions of the board if you set them yourself from inside the code.

	\item Now, have your program change the board based on input from the command line.

	\item Start with Player 1. Ask them for a move, and read what choice they make. Then update that part of the board to hold an 'x'. 

	\item Print the board before and after.
	
	\begin{42console}
	?> tictac.rb
	Welcome to Intergalactic Tic Tac Toe!
	   |   |   
	-----------
	   |   |   
	-----------
	   |   |   

	Player 1, it's your turn. Which square?
	 0 | 1 | 2 
	-----------
	 3 | 4 | 5 
	-----------
	 6 | 7 | 8 

	 2
	 Interesting choice, Player 1. 
	   |   | x 
	-----------
	   |   |   
	-----------
	   |   |   
	\end{42console}

\end{itemize}

\chapter{Looping}

Awesome! You know how to print the gameboard and store whether there is an x or an o in each square. You also have been able to receive a move from a player and make it real.

\begin{itemize}

	\item In the main body of your program, add a loop which repeatedly asks Player 1 and Player 2 for their moves.

	\item It should alternate between the two players and print the board after each move.

	\item For now, the loop can be infinite.

\end{itemize}

\chapter{Validity}

Validity checking - one of the most important things you will ever need to remember as a programmer.

\begin{itemize}

	\item To prevent cheating, insert a check to make sure that a players' move is valid.

	\item A move is not valid if it is outside the game board.

	\item A move is not valid if someone's x or o is already played there.

\end{itemize}

\chapter{Did I Win?}

Looking good! Kudos that you have made it this far. 

\begin{itemize}

	\item Create a method in your program which determines if anyone has won yet.

	\item This may be the most complex part of the program so far! Remember to check for horizontal, vertical, and diagonal lines.

	\item Add this check to your loop so that we look for a win after every move.

	\item Print a congratulatory message when you do find a win!

\end{itemize}

%******************************************************************************%
%                                                                              %
%                               End of document                                %
%                                                                              %
%******************************************************************************%

\end{document}
