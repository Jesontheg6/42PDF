
% vim: set ts=4 sw=4 tw=80 noexpandtab:
%******************************************************************************%
%                                                                              %
%                   Parseltongue_04.tex                                        %
%                   Made by: 42 staff                                          %
%                                                                              %
%******************************************************************************%

\documentclass{42-en}

%******************************************************************************%
%                                                                              %
%                                   Prologue                                   %
%                                                                              %
%******************************************************************************%

\begin{document}

\title{Parseltongue Piscine - Day04}
\subtitle{keep practicing and build more complex projects}

\member {Kai}{kai@42.us.org}

\maketitle

\tableofcontents

%Initialisation des headers d'exercices

\newpage

\bigskip

\centerline{\includegraphics[width=150mm]{images/dont_panic.jpeg}}

\centerline{\texttt{Eat, Sleep, Code, Repeat.}}

%******************************************************************************%
%                                                                              %
%                                 Don't Panic                                  %
%                                                                              %
%******************************************************************************%

\chapter{Don't Panic!}

Confused about how to begin? Not sure what the PDF means?\\

Don't worry, we are not perfect :)\\

And the PDFs are challenges - not walkthroughs. Team up with your group, your partner,
your mentor and your other peers to decipher what to do.\\

You are the master of your own destiny! Go forward and code the world !

%******************************************************************************%
%                                                                              %
%                              Coding Guidelines                               %
%                                                                              %
%******************************************************************************%

\chapter{Format your Code}

Each 42 challenge you turn in must adhere to the following format:

\begin{42pycode}
#!/usr/bin/env python3

# Write your name at the top, and any helpful comments you have for people
# running your program.
# By <userid>

import sys

def function_a:
    # code

def function_b:
    # code

def main(argv):
    # main method
    function_a
    function_b

main(sys.argv)
\end{42pycode}

or:\\

\begin{42rbcode}
#!/usr/bin/env ruby

# Write your name at the top, and any helpful comments you have for people
# running your program.
# By <userid>

def function_a
 # code
end

def function_b
 # code
end

def main(ARGV)
 # main method
 function_a
 function_b
end

main(ARGV)
\end{42rbcode}

\begin{itemize}
	\item Always begin with the "\#!/usr/bin/env ruby" statement. This tells your terminal to run the program using Ruby. In python, the first line is "\#!/usr/bin/env python3".
	\item Always add a comment stating what this program is for, some hints to help others use or understand it, and your name or intra ID.
	\item Do not write any code outside of functions except for one line, at the end of your program, which calls the main() function.
\end{itemize}

\startexercices

%******************************************************************************%
%                                                                              %
%                              Student Directory                               %
%                                                                              %
%******************************************************************************%

\chapter{Exercise 0: Student Directory}

\extitle{Student Directory}
\exfiles{00\_phonebook.rb or 00\_phonebook.py}
\exnotes{Use the names.txt file provided on the project page. Ruby \href{https://ruby-doc.org/core-2.4.2/Hash.html}{Hash} Python \href{https://docs.python.org/2/tutorial/datastructures.html\#dictionaries}{Dictionary}, \href{http://www.pythonforbeginners.com/dictionary/how-to-use-dictionaries-in-python}{More Dictionaries}, \href{https://www.youtube.com/watch?v=daefaLgNkw0}{Key} Value Pairs}
\makeheaderfiles

Using the attached file names.txt, store the information in a hash or dictionary where first names are associated with last names.\\

Use your hashtable to identify which first names are shared by more than one student, mentor or admin in h2s. Print out each first name that repeats in the set followed by an array of the last names associated with that first name. Then do the same thing with last names.

\begin{42console}
	?> python 00_phonebook.py
	** Shared First Names! **
	Elliot (2): [Tregoning, VanHeuman]

	** Shared Last Names **
	Kardashian (4): [Khloe, Kim, Kourtney, Rob]
\end{42console}

\nextexercice
\newpage

%******************************************************************************%
%                                                                              %
%                                   State Capitols                             %
%                                                                              %
%******************************************************************************%

\chapter{Exercise 1: State Capitols}
\extitle{State Capitols}
\exfiles{01\_capitols.rb or 01\_capitols.py}
\exnotes{Use the capitols.txt file provided on the project page. You do not need to turn that file in, but you can include it in your repository. Ruby \href{https://alvinalexander.com/blog/post/ruby/how-process-line-text-file-ruby}{Text files in Ruby} Python \href{https://docs.python.org/3/tutorial/inputoutput.html}{Input/Output}, \href{https://www.youtube.com/watch?v=-Lu8VgYSkNQ}{Text Files in Python}, Python \href{https://www.python-course.eu/python3_dictionaries.php}{dictionaries}}
\makeheaderfiles

\begin{itemize}

\item Create a script \texttt{01\_capitols.py} which reads in the provided comma-delimited file of US States and capitals and stores this information in a hashtable.
\item Next, on an infinite loop, print "Ready: " and wait for the user to enter the name of a state or capital. For each query print out the associated capital or state and go back to Ready state.
\item The program exits when the user types "Done". If the input is invalid, answer "nil".

\begin{42console}
	?> python 01_capitols.py capitals.txt
	Ready: Arizona
	Phoenix
	Ready: Montana
	Helena
	Ready: MacaroniAndCheese
	nil
	Ready: Pierre
	South Dakota
	Ready: Done
	?>
\end{42console}

\end{itemize}

\nextexercice
\newpage

%******************************************************************************%
%                                                                              %
%                                   101010                                     %
%                                                                              %
%******************************************************************************%

\chapter{Exercise 2: DIY 101010}

\exfiles{02\_allyourbase.rb or 02\_allyourbase.py}

\makeheaderfiles

Write a decimal to binary converter of your very own.\\

Do not use built in string formatting functions; you must write the mathematical logic for converting between bases, and print out the result.\\

\begin{42console}
	?> python 02_allyourbase.py 94555
	10111000101011011
	270533
	1715B
	?>
\end{42console}

\end{document}
