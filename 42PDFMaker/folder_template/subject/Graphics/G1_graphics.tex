% vim: set ts=4 sw=4 tw=80 noexpandtab:
%******************************************************************************%
%                                                                              %
%                   G1_Graphics.tex                                            %
%                   Made by: Kai                                               %
%                                                                              %
%******************************************************************************%

\documentclass{42-en}

%******************************************************************************%
%                                                                              %
%                                   Prologue                                   %
%                                                                              %
%******************************************************************************%

\begin{document}

\title{Game Design 1: Graphics with Processing}
\subtitle{The World is in Color Now}

\member {Kai}{kai@42.us.org}

\summary
{
	The second week is all about designing beautiful things with the graphics library found at processing.org.
}

\maketitle

\tableofcontents

%******************************************************************************%
%                                                                              %
%                               Getting Started                                %
%                                                                              %
%******************************************************************************%

\chapter{Processing Tutorials}

You should start with the tutorials at \href{http://py.processing.org/tutorials/}{py.processing.org}.\\

\begin{enumerate}

	\item Install Processing, an application where you can write and run your graphics code.
	\item Set up Python mode inside of processing.
	\item Complete the following tutorials:
	\item \href{http://py.processing.org/tutorials/gettingstarted/}{"Getting Started"}
	\item \href{http://py.processing.org/tutorials/overview/}{"Overview"}
	\item \href{http://py.processing.org/tutorials/drawing/}{"Coordinate System and Shapes"}
	\item \href{http://py.processing.org/tutorials/color/}{"Color"}
	\item \href{http://py.processing.org/tutorials/interactivity/}{"Interactivity"}

\end{enumerate}

These will give you an understanding of the tools you'll use to create a graphic based game. The most important concepts to grasp are the coordinate system, and the color system, as well as how to detect mouse clicks and respond to them. Go ahead and complete others when you think they will be useful.


%******************************************************************************%
%                                                                              %
%                               Getting Started                                %
%                                                                              %
%******************************************************************************%

\chapter{Hackathon!}

Meet with your group and brainstorm what kind of game you would like to create. You can re-use the logic from the first week, or build a new one - but don't let one person take control of the whole project. Find something that everyone is interested in.\\

In a game hackathon, coding is not the only skill needed. How will you make your game beautiful and well-designed? Maybe one or a few people in your group can take initiative in sketching what it will look like.

Here, as a reminder, is the same list from last time:

\begin{itemize}

	\item Mastermind
	\item Tic-Tac-Toe
	\item 3d Tic-Tac-Toe
	\item Connect 4
	\item Sudoku
	\item Minesweeper
	\item Hangman
	\item Word Search
	\item Game of Life
	\item Cryptographic Cipher Encoder/Decoder
	\item Flash cards quiz game
	\item Base converter
	\item Metric to Imperial converter
	\item Matching game (guess two cards on the board until you find pairs)
	\item Go Fish
	\item War
	\item Othello

\end{itemize}

Beware of choosing anything with complex logic and physics like a shooter game. It's best to build something straightforward first, and make sure it turns out well. We'll build games with physics after studying object-oriented programming, in Game Design 2.\\

There will be a showcase and contest on Friday! Try to build a different game than everyone else and make it shine with your own creativity.\\

\end{document}