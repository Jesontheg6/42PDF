% vim: set ts=4 sw=4 tw=80 noexpandtab:
%******************************************************************************%
%                                                                              %
%                   GraphicsWithProcessing.tex                                 %
%                   Made by: Kai                                               %
%                                                                              %
%******************************************************************************%

\documentclass{42-en}

%******************************************************************************%
%                                                                              %
%                                   Prologue                                   %
%                                                                              %
%******************************************************************************%

\begin{document}

\title{Graphics with Processing: Overview}
\subtitle{Learn to use an excellent graphics and visualization tool.}

\member {Kai}{kai@42.us.org}

\summary
{
	This branch of the progam is all about designing beautiful things and illustrating
	data or mathematic concepts with the flexible tools found at processing.org.
}

\maketitle

\tableofcontents

%******************************************************************************%
%                                                                              %
%                               Getting Started                                %
%                                                                              %
%******************************************************************************%

\chapter{Getting Started}

You will need to \href{https://processing.org/download/}{download} the Processing application. Open the installer and move the Processing icon to your desktop or to a folder that you create for yourself. For example, in my home directory I have the standard folder "Applications" but also a custom folder called "MyApplications". 42 will not give you an administrator password to install things on the computers, but you can often add programs easily this way.\\

There is an extension which can allow us to write code for Processing in Ruby, but it is not currently supported on our lab computers. I will update the PDF with install instructions when it is. For now, you can use Processing with Java, Python, or Javascript syntax.\\

Double click on the Processing icon to open their user interface. For these projects, you can write code directly into the Processing text box, and then click the Play/Stop buttons to execute it. Make sure to save your work in a file on your computer, though.\\

In the top right corner there is a dropdown menu which says "Java" by default. Click the drop-down arrow and choose "Add Mode...". Then, select the mode you prefer and click Install to make it available. Each mode has the same abilities but uses a different syntax.\\

Use the drop down menu in the top right corner to switch to your preferred mode before you start typing.


%******************************************************************************%
%                                                                              %
%                               Drawing Things                                 %
%                                                                              %
%******************************************************************************%

\chapter{Drawing Things}

Processing wrote their own introduction as part of Computer Science Education week. Why not start by working through that? Follow along from \href{http://hello.processing.org/}{Processing's Hello World} and type their commands into your Processing application to start learning how it works.\\

There are many other tutorials on various topics on the Processing website.\\

\begin{itemize}
	\item Java syntax: \href{https://processing.org/tutorials/}{Tutorials}
	\item Python syntax: \href{http://py.processing.org/tutorials/}{Tutorials}
	\item Javascript syntax and embedding graphics within an HTML page: \href{https://p5js.org/learn/}{Introduction to p5.js}
\end{itemize}

Within this overview project, we have three specific challenges for you.
\begin{itemize}
	\item Animate the Game of Life, a classic computer science simulation. There are many ways you can do this and features you can add. At the basic level, you hard-code a beginning configuration and run the simulation from there. At an advanced level, allow the end user to set the beginning configuration and then press play.
	\item Prove your trigonometry skills by animating the rotation of a 3D object in space.
	\item Explore the mathematical wonderland of fractals and imaginary numbers, starting with the classic Mandelbrot and Julia sets. 
\end{itemize}

You can get credit for a wildcard project of your own, and if it is something that everyone would like then we'll add it to the list! :)\\

How about a data visualization project.... like \href{http://me.whererabbitsthink.com/index.php/2017/09/11/facebook-friends/}{this?}


\end{document}