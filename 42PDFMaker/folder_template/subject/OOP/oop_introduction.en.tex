%******************************************************************************%
%                                                                              %
%                  sample.en.tex for LaTeX                                     %
%                  Made by : Michael Lu (mlu@student.42.fr)                    %
%                                                                              %
%******************************************************************************%

\documentclass{42-en}


%******************************************************************************%
%                                                                              %
%                                    Header                                    %
%                                                                              %
%******************************************************************************%
\begin{document}



                           \title{OOP\_Part 1}
                          \subtitle{Essentials of Object Oriented Programming}
                       \member{Michael Lu}{mlu@student.42.fr}
\summary {
  This project will help you learn the essentials of objective oriented programming.
}

\maketitle

\tableofcontents

%******************************************************************************%
%                                                                              %
%                                  Foreword                                    %
%                                                                              %
%******************************************************************************%
\chapter{Foreword}

	Did you know I love cooking and cooking is a great way to learn stuff?\\

	Cooking teaches you a lot of skills that is beneficial to you. Regardless
	if you prefer your mom's home cooking or maybe your dad's barbecue, but
	have you ever tried following a recipe and learning it yourself?\\

	If you ever start learning how to cook you will soon realize you need a
	couple of things first. You need measuring tools, some kind of hot plate,
	a way to cut or prep ingredients. Each of these objects are their own
	entity but they work together to provide you a delicious meal.\\

	Object oriented programming is very similar to this concept. Hah! Bet you
	thought I wouldn't bring up programming eh? Just like cooking, you will
	be creating objects in objected oriented programming (I wonder why it's called
	that), and learning how to utilize them to help you create some cool stuff.\\

	\begin{center}
		\includegraphics[width=0.6\textwidth]{images/thonking.jpg}
	\end{center}

%******************************************************************************%
%                                                                              %
%                                  Goals                                       %
%                                                                              %
%******************************************************************************%
\chapter{Goals}

	The goal of this project is to introduce you to the basics object oriented programming.
	By the end of this project you should know how to:\\
	
	\begin{itemize}
		\item Create classes
		\item Initiate an instance of a class
		\item Assign class methods and attributes
		\item Use variables specific to each instance
		\item Design classes that inherit from each other.
	\end{itemize}
	 
	You will be exploring a fundamental topic of object oriented programming
	so take advantage of all the resources including articles, videos, your neighbor, StackOverflow and so forth. There are many tutorials on classes and inheritance.


%******************************************************************************%
%                                                                              %
%                             General instructions                             %
%                                                                              %
%******************************************************************************%
\chapter{General instructions}

	\begin{itemize}
		\item This project will only be corrected by actual human beings.
		You are therefore free to organize and name your files as you wish,
		although you must respect some requirements listed below.
		\item You must follow the exercise details and instruction clearly
		\item You must turn in all the requested files
		\item Ask your peers, mentor, slack or anywhere else if you need
		any help, and make sure to have fun!
	\end{itemize}

\startexercices

%******************************************************************************%
%                                                                              %
%                                    ex00                                      %
%                                                                              %
%******************************************************************************%
\chapter{Exercise 00}

\extitle{ex00: Your first class}
\exfiles{main.(rb/py), first\_class.(rb/py)}

\makeheaderfiles

Make your first class (a class named FirstClass in a file named first\_class) with a constructor that says "Hello World". Instanciate the class inside of a main function located in main.(rb/py).

\begin{42console}
	?> ruby first_class.py
	Hello World
	?>
\end{42console}

\hint{Google classes or check ft\_arena or ft\_boardgame tutorial videos}

\hint{if __name__ == "__main__":}
\hint{Ruby: require\_relative first\_class. Python: from . import first\_class}
\nextexercice
%******************************************************************************%
%                                                                              %
%                                    ex01                                      %
%                                                                              %
%******************************************************************************%
\chapter{Exercise 01}

\extitle{ex01 : Your second class and first inheritance}
\exfiles{main.(rb/py), first\_class.(rb/py), second\_class.(rb/py)}

\makeheaderfiles

Make your second class (a class named SecondClass in a file named second\_class) that will inherit from the first class. Its constructor should directly call the first class constructor that says "Hello World". You must instantiate the second class only in your main.

\begin{42console}
	?> ruby main.rb
	Hello World
	?>
\end{42console}

\hint{Google class inheritance or check ft\_arena or ft\_boardgame tutorial videos}
\nextexercice
%******************************************************************************%
%                                                                              %
%                                    ex02                                      %
%                                                                              %
%******************************************************************************%
\chapter{Exercise 02}

\extitle{ex02 : Your first parameter and passing parameter}
\exfiles{main.(rb/py), first\_class.(rb/py), second\_class.(rb/py)}

\makeheaderfiles

You now need your second class to take a parameter "name" (which will be your login name) and pass it into the first class which will display "Hello ". You can hard-code your login name into the program without calling input() or gets() to fetch it. You must instantiate the second class only in your main.

\begin{42console}
	?> ruby main.rb
	Hello mlu
	?>
\end{42console}

\hint{Google how to send parameter into classes or check ft\_arena or ft\_boardgame tutorial videos}
\nextexercice
%******************************************************************************%
%                                                                              %
%                                    ex03                                      %
%                                                                              %
%******************************************************************************%
\chapter{Exercise 03}

\extitle{ex03 : Your first method}
\exfiles{main.(rb/py), first\_class.(rb/py), second\_class.(rb/py)}

\makeheaderfiles

You now need to create a method inside your first class called say\_hello which will take the name from the constructor and print out "Hello ". When the method is called pring out a sentance stating that it has been called. You must instantiate the second class only in your main.

\begin{42console}
	?> ruby main.rb
	Method say_hello in FirstClass is called
	Hello mlu
	?>
\end{42console}

\hint{Google class methods or check ft\_arena or ft\_boardgame tutorial videos}
\nextexercice
%******************************************************************************%
%                                                                              %
%                                    ex04                                      %
%                                                                              %
%******************************************************************************%
\chapter{Exercise 04}

\extitle{ex04 : Your second method}
\exfiles{main.(rb/py), first\_class.(rb/py), second\_class.(rb/py)}

\makeheaderfiles

You now need to create a method inside your second class called roll\_dice which will randomly generate a number from 1 to 6. Roll\_dice will then call the Hello method in its parent class and pass it the random number. You should see an output of "Hello <username>, your number is ". Remember, every method your write should print something to announce it has been called. You must instantiate the second class only in your main.

\begin{42console}
	?> ruby main.rb
	Method roll_dice in SecondClass called
	Method hello in FirstClass is called
	Hello mlu, your number is 2
	?>
\end{42console}

\hint{Google class methods/methods interaction or check ft\_arena or ft\_boardgame tutorial videos}
\nextexercice
%******************************************************************************%
%                                                                              %
%                                    ex05                                      %
%                                                                              %
%******************************************************************************%
\chapter{Exercise 05}

\extitle{ex05 : Your first class variables, and more methods!}
\exfiles{main.(rb/py), first\_class.(rb/py), second\_class.(rb/py)}

\makeheaderfiles

Your second class will now take in a second parameter, a string called "hobby", when it is instantiated. Store the hobby in an instance variable. You will write a method called get\_hobby in your second class that returns this variable. In your main you need to print out "Your hobby is " where must be returned from the get\_hobby method. You must instantiate the second class only in your main.

\begin{42console}
	?> ruby main.rb
	Method roll_dice in SecondClass called
	Method hello in FirstClass is called
	Hello mlu, your number is 5
	Your hobby is being lazy
	?>
\end{42console}

\hint{Read about the difference between class variables and instance variables. What is an instance of a class?}


%******************************************************************************%
\end{document}
