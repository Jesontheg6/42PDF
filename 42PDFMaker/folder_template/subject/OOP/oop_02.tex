%******************************************************************************%
%                                                                              %
%                  oop_02.tex for LaTeX                                        %
%                  Created on : Tue Mar 10 13:27:28 2015                       %
%                  Made by : David "Thor" GIRON <thor@42.fr>                   %
%                                                                              %
%******************************************************************************%

\documentclass{42-en}


%******************************************************************************%
%                                                                              %
%                                    Header                                    %
%                                                                              %
%******************************************************************************%
\begin{document}



       \title{Object Oriented Programming, Part 2 - Linked Lists}
      \subtitle{The Archnemesis of Arrays}
   \member{tingo -}{tingo@student.42.us.org}
   \member{jchung -}{jchung@student.42.us.org}

\summary {
  This is an introduction to \texttt{Linked Lists}.
}

\maketitle

\tableofcontents


%******************************************************************************%
%                                                                              %
%                                  Foreword                                    %
%                                                                              %
%******************************************************************************%
\chapter{Foreword}
\begin{center}
	\textbf{Rick Astley - Never Gonna Give You Up}
\end{center}
We're no strangers to love\\
    You know the rules and so do I\\
    A full commitment's what I'm thinking of\\
    You wouldn't get this from any other guy\\
    I just wanna tell you how I'm feeling\\
    Gotta make you understand\\
    \\
    Never gonna give you up\\
    Never gonna let you down\\
    Never gonna run around and desert you\\
    Never gonna make you cry\\
    Never gonna say goodbye\\
    Never gonna tell a lie and hurt you\\
    \\
    We've known each other for so long\\
    Your heart's been aching but you're too shy to say it\\
    Inside we both know what's been going on\\
    We know the game and we're gonna play it\\
    And if you ask me how I'm feeling\\
    Don't tell me you're too blind to see\\
    \\
    Never gonna give you up\\
    Never gonna let you down\\
    Never gonna run around and desert you\\
    Never gonna make you cry\\
    Never gonna say goodbye\\
    Never gonna tell a lie and hurt you\\
    Never gonna give you up\\
    Never gonna let you down\\
    Never gonna run around and desert you\\
    Never gonna make you cry\\
    Never gonna say goodbye\\
    Never gonna tell a lie and hurt you\\
    Never gonna give, never gonna give\\
    \\
    (Give you up)\\
    (Ooh) Never gonna give, never gonna give\\
    (Give you up)\\
    \\
    We've known each other for so long\\
    Your heart's been aching but you're too shy to say it\\
    Inside we both know what's been going on\\
    We know the game and we're gonna play it\\
    I just wanna tell you how I'm feeling\\
    Gotta make you understand\\
    \\
    Never gonna give you up\\
    Never gonna let you down\\
    Never gonna run around and desert you\\
    Never gonna make you cry\\
    Never gonna say goodbye\\
    Never gonna tell a lie and hurt you\\
    Never gonna give you up\\
    Never gonna let you down\\
    Never gonna run around and desert you\\
    Never gonna make you cry\\
    Never gonna say goodbye\\
    Never gonna tell a lie and hurt you\\
    Never gonna give you up\\
    Never gonna let you down\\
    Never gonna run around and desert you\\
    Never gonna make you cry

%******************************************************************************%
%                                                                              %
%                                 Introduction                                 %
%                                                                              %
%******************************************************************************%
\chapter{Introduction}
    Linked lists are the foundation for
    more advanced data structures and are commonly used in place of arrays
    for data storage due to their ability to store information in
    non-contiguous blocks of memory. This makes linked lists a very powerful
    tool for parallel computation.\\
    \\
    There are many uses for the different variations of linked lists.
    For example, many games will store cycling character animations in a circular linked
    list or a circular array. Many operating systems line up processes and 
    jobs in a queue. A simulation of genetic mutations like removing and swapping base pairs in DNA strands can best
    be organized by a doubly linked list.\\
    \\
    Within the 42 curriculum, the projects in graphics branch frequently utilize
    linked lists to store coordinates of points for an object in a map or
    rendering. In many Unix projects, making a process and job queue for
    operating systems is essential to keep the computer running without a crash.



%******************************************************************************%
%                                                                              %
%                                  Goals                                       %
%                                                                              %
%******************************************************************************%
\chapter{Goals}
    This project lays groundwork for all of the data structures you will learn
    in the following days. Today, you will learn how to implement your very own
    data structure. You will be responsible for creating the basic functions
    that allow a data structure to access values, add values, remove values, and
    modify itself in fun and interesting ways.\\
    \\
    This project is a simple gateway into the larger world of Data Structures.

%******************************************************************************%
%                                                                              %
%                             General instructions                             %
%                                                                              %
%******************************************************************************%
\chapter{General instructions}
    The following exercises are designed such that written functions are
    to be turned in as standalone functions; they will not be included
    in the class for LinkedLists but will still perform standardized
    operations. All exercises should be written in Python and no external
    function calls should be made except for the first question; only the
    functions supplied in the classes provided to you should be needed.\\
    \\
    Both a \texttt{Node} and \texttt{SinglyList} class have been given to
    you for use. The \texttt{SinglyList} class has an in-class iterator defined to
    traverse a list, but you are free to iterate through the list yourself :)\\
    \\
    If a function requires the use of the Node or SinglyList class, do not 
    include that code in your file submission. If you use a previous solution
    please make sure to include that function in the file submission.

\newpage
    \section{Node Class}

        \begin{42pycode}
class Node(object):
    def __init__(self, content):
        if content is None:
            raise ValueError('Node must have content')
        self.c = content
        self.n = None
    
    @property
    def content(self):
        return self.c

    @content.setter
    def content(self, val):
        self.c = val
    
    @property
    def next(self):
        return self.n
    
    @next.setter
    def next(self, val):
        self.n = val
\end{42pycode}

    \section{SinglyList Class}

        \begin{42pycode}
class SinglyList(object):
    def __init__(self):
        self.h = None

    def __iter__(self):
        current = self.head
        while current:
            yield current
            current = current.next

    @property
    def head(self):
        return self.h

    @head.setter
    def head(self, val):
        self.h = val

    def isEmpty(self):
        return self.head == None

    def add_head(self, node):
        if self.isEmpty():
            self.head = node
        else:
            node.next = self.head
            self.head = node
    \end{42pycode}

\startexercices

\chapter{Exercise\exercicenumber: Print All Nodes in a List}

\extitle{Print All Nodes in a List}
\exnumber{\exercicenumber}
\exfiles{print\_list.py}
\exforbidden{Everything except print() :D}
\exnotes{n/a}

\makeheaderfiles
    Understanding how to traverse a linked list is important to mastering
    its concept. Given the head node of a singly linked list, print out all
    the items in the list.\\
    \\
    Your function will be defined as such:\\
    \texttt{print\_list(list\_head)}\\
    \\
    \textbf{Input Format}

    Your function will take the head node of a list.\\
    \\
    \textbf{Output Format}

    Nothing special. Just print the items of the list in the order they appear.

\nextexercice

\chapter{Exercise\exercicenumber: Add an Item to the End of a List}

\extitle{Add an Item to the End of a List}
\exnumber{\exercicenumber}
\exfiles{add\_tail.py}
\exforbidden{Everything :D}
\exnotes{There is a similar function in the SinglyList class to reference :)}

\makeheaderfiles
    You're given the pointer to the head node of a singly linked list and
    the value of a node to add to the list. Create a new node with the given
    value. Insert this node at the tail of the linked list and return the head
    node after the insertion.\\
    \\
    Your function will be defined as such:\\
    \texttt{add\_tail(list\_head, val)}\\
    \\
    \textbf{Input Format}

    Your function will take the head node of a list and a value.\\
    \\
    \textbf{Output Format}

    There will be no return, but the list must now cotain a new node with the specified value at the end of it.

\nextexercice

\chapter{Exercise\exercicenumber: Remove an Item From a List}

\extitle{Remove an Item From a List}
\exnumber{\exercicenumber}
\exfiles{remove.py}
\exforbidden{Everything :D}
\exnotes{If the list is empty, head will be null. The list will not contain duplicates.}

\makeheaderfiles
    You're given the pointer to the head node of a singly linked list and
    the value of a node to delete from the list. Delete the node with the given
    value if it exists.\\
    \\
    Your function will be defined as such:\\
    \texttt{remove(list\_head, val)}\\
    \\
    \textbf{Input Format}

    Your function will take the head node of a list and a value.\\
    \\
    \textbf{Output Format}

    There will be no return, but the list must no longer contain the node with that value.

\nextexercice

\chapter{Exercise\exercicenumber: Cycle Detection}

\extitle{Cycle Detection}
\exnumber{\exercicenumber}
\exfiles{has\_cycle.py}
\exforbidden{Everything :D}
\exnotes{If the list is empty, head will be null.}

\makeheaderfiles
    In a turn-based multiplayer game, a linked list can be used to cyclically
    repeat player order by having the last player's \texttt{next} reference the
    first player. Check that a given linked list cycles or not.\\
    \\
    Your function will be defined as such:\\
    \texttt{has\_cycle(list\_head)}\\
    \\
    \textbf{Input Format}

    Your function will take the head node of a list.\\
    \\
    \textbf{Output Format}

    It will return True of there is a cycle and False if there is no cycle.

\nextexercice

\chapter{Exercise\exercicenumber: Merge Two Lists}

\extitle{Merge Two Lists}
\exnumber{\exercicenumber}
\exfiles{merge.py}
\exforbidden{Everything :D}
\exnotes{All trains have at least one car, and no two cars have the same weight.}

\makeheaderfiles
    Two cargo trains arrive in the trainyard and their cargo needs to be
    consolidated into a single train set before it can depart. Both trains
    were organized with the heaviest car in the front of the train
    descending to the lightest car in the back.\\
    \\
    Your function will be defined as such:\\
    \texttt{merge(train1, train2)}\\
    \\
    \textbf{Input Format}

    Your function will take two list heads, \textit{train1 and train2}.\\
    \\
    \textbf{Output Format}

    Return the head of the new merged train.

\nextexercice

\chapter{Exercise\exercicenumber: Sort a Linked List}

\extitle{Sort a Linked List}
\exnumber{\exercicenumber}
\exfiles{sort\_asc.py}
\exforbidden{Everything :D}
\exnotes{There is no limitation on which sort to implement for this exercise. You will find some sorts are easier to work into a linked list than others...}

\makeheaderfiles
    Understanding how to organize information in a list is important. Implement
    a sorting algorithm that organizes a linked list with numbers in
    \textbf{ascending} order.\\
    \\
    Your function will be defined as such:\\
    \texttt{sort\_asc(unsorted\_list)}\\
    \\
    \textbf{Input Format}

    Your function will take the head node of an unsorted list.\\
    \\
    \textbf{Output Format}

    Sort the list in function, but return nothing.

\chapter{Bonus part}
    Now that you've had a chance to get your feet wet with the linked list, let's
    revisit a previous problem. The traiyard has run into some more trouble.
    See if you can help them.

\warn{This will only count if and only if you have completed the previous excercises correctly. Go back now to make sure.}

\nextexercice

\chapter{Exercise\exercicenumber: Trainyard Revisited}

\extitle{Trainyard Revisited}
\exnumber{\exercicenumber}
\exfiles{trainyard.py}
\exforbidden{Everything :D}
\exnotes{All trains have at least one cart, and two carts \textbf{may} have the same weight.}

\makeheaderfiles
    Two more cargo trains arrive in the trainyard and need to be merged before
    departure. These two trains are no longer organized by weight from heaviest
    to lightest prior to the merge. The two lightest cars must also stay behind
    at the trainyard for inspection and leave with the next incoming train.
    Take into account cars with equal weights.\\
    \\
    Your function will be defined as in excercise 04.\\
    \\
    \textbf{Input Format}

    \textit{Same as 04}\\
    \\
    \textbf{Output Format}

    \textit{Same as 04}, but missing the lightest 2 and sorted.

%******************************************************************************%
%                                                                              %
%                           Turn-in and peer-evaluation                        %
%                                                                              %
%******************************************************************************%
\chapter{Turn-in and peer-evaluation}

    This project will be evaluated by your peers and the moulinette. Submit to
    the \texttt{GIT} repository as usual. Only work in your repository will be
    graded. Make sure you have the \textit{exact} function names as shown in
    this document. Only the specified functions will be looked at and graded.
    Remember, the bonus will only be evaluated if \textit{all} the previous
    excercises are correct.

\end{document}
