
% vim: set ts=4 sw=4 tw=80 noexpandtab:
%******************************************************************************%
%                                                                              %
%                   Day01.tex                                                    %
%                   Made by: 42 staff                                          %
%                                                                              %
%******************************************************************************%

\documentclass{42-en}

%******************************************************************************%
%                                                                              %
%                                   Prologue                                   %
%                                                                              %
%******************************************************************************%

\begin{document}

\title{Introduction to Programming - 42}
\subtitle{Day 01}

\member {Kai}{kai@42.us.org}
\member {Gaetan}{gaetan@42.us.org}

\summary
{
This document is the subject of the day 01 of the introduction to programming piscine.
}

\maketitle
\tableofcontents

%Initialisation des headers d'exercices
\daypiscine{01}

%******************************************************************************%
%                                                                              %
%                                Guidelines                                    %
%                                                                              %
%******************************************************************************%

\chapter{Guidelines}

\begin{itemize}

  \item Corrections will take place in the last hour of the day. Each person will correct another person according to the peer-corrections model.
 
  \item Questions? Ask the neighbor on your right. Next, ask the neighbor on your left.
  
  \item Read the examples carefully. The exercises might require things that are not specified in the subject...

  \item Your reference manual is called Google / "Read the Manual!" / the Internet / ...

\end{itemize}

\newpage



%******************************************************************************%
%                                                                              %
%                                Preamble                                      %
%                                                                              %
%******************************************************************************%

\chapter{Preamble}

Here is what \texttt{Wikipédia} has to say about the otter:

\begin{verbatim}
Otters are animals that live near and around water. They are a part of the animal 
family Mustelid. A group of otters is called a "romp", because they play together 
and are energetic. They live in nests called holts. There are many different kinds 
of otters. Some live near rivers, some in the sea (Sea Otters). Otters live on
every continent except for Australia and Antarctica.

Otters are carnivores; they must eat a lot of meat to live, so they spend 3 to 5 
hours every day fishing and hunting. They can die of hunger more quickly than most
animals. They eat fish, crayfish, crabs, and frogs. They will eat any small
animals and birds they can catch. Much of their food is found in water. They dive
in rivers, lakes and streams until they can find a suitable animal to eat, which
they then chase. Once they catch it, they bring it to the top of the water, where
they eat it. If the animal has a hard shell, otters can use a rock as a tool to
break open the shell.

Otters are playful and energetic. They can be seen sliding down hills and slopes,
as well as chasing other otters for fun. Some kinds of otters live in groups,
while others are almost always alone. Because they spent so much time in cold
water, they have to groom their fur often to stop themselves from freezing.

Otters talk to other animals and otters with whistles, growls, chuckles and
screams, as well as chirps, squeals, and some kinds of otters might even purr.
They also leave their scent on plants to mark their territory.

Sea otters (Enhydra lutris) are classified as marine mammals. They live along the
Pacific coast of North America. Their historic range included shallow waters of
the Bering Strait. It also included Kamchatka, and as far south as Japan. They
have a rich fur for which humans hunted them almost to extinction. They frequently
carry a rock in a pouch under their forearm. They use this to smash open shells.
This makes them one of the few animals that use tools. Although once near extinction,
they have begun to spread again, from remnant populations in California and Alaska.

\end{verbatim}

It's a cute otter.

\startexercices

%******************************************************************************%
%                                                                              %
%                                  42                                          %
%                                                                              %
%******************************************************************************%

\chapter{Exercise \exercicenumber: 42}

\extitle{How to display 42}
\exnumber{\exercicenumber}
\exscore{2}
\exfiles{42.rb}
\exauthorize{All}

\makeheaderfiles

\begin{itemize}

\item Create a script \texttt{42.rb} that, when executed, displays "42" followed by a newline.

\begin{42console}
	?> ./42.rb | cat -e
	42$
	?>
\end{42console}

\end{itemize}

\hint{Does it sound simple? That's normal.}
\hint{Ruby is a language meant to read like plain English - easier for us!}

%******************************************************************************%
%                                                                              %
%                                   Name                                       %
%                                                                              %
%******************************************************************************%

\chapter{Exercise \exercicenumber: name}

\extitle{Display a name}
\exnumber{\exercicenumber}
\exscore{2}
\exfiles{name.rb}
\exauthorize{All}

\makeheaderfiles

\begin{itemize}

\item Create a script \texttt{name.rb} in which you define a first\_name variable and a last\_name variable, initialized with your first and last names respectively, and then display them followed by a newline.

\begin{42console}
	?> ./name.rb | cat -e
	Arthur Dent$
	?>
\end{42console}

\end{itemize}


%******************************************************************************%
%                                                                              %
%                                   Name++                                     %
%                                                                              %
%******************************************************************************%

\chapter{Exercise \exercicenumber: name++}

\extitle{Display a name, better}
\exnumber{\exercicenumber}
\exscore{2}
\exfiles{name.rb}
\exauthorize{All}

\makeheaderfiles

\begin{itemize}

\item Modify the script name.rb: still do the same thing, but this time your first and last name should be concatenated and assigned to a third variable.

\begin{42console}
	?> ./name.rb | cat -e
	Ford Prefect$
	?>
\end{42console}

\end{itemize}

\hint{Google ruby strings.}

%******************************************************************************%
%                                                                              %
%                              What's your name                                %
%                                                                              %
%******************************************************************************%

\chapter{Exercise \exercicenumber: what's your name}

\extitle{What's your name?}
\exnumber{\exercicenumber}
\exscore{2}
\exfiles{whatsyourname.rb}
\exauthorize{All}

\makeheaderfiles

\begin{itemize}

\item Create a script whatsyourname.rb first asks the user to enter their first name, then their last name, and finally displays both.

\begin{42console}
	?> ./whatsyourname.rb
	Hey, what's your first name ? : Arthur
	And your last name ? : Dent
	Well, pleased to meet you Arthur Dent.
	?>
\end{42console}

\end{itemize}

\hint{Google gets, chomp.}


%******************************************************************************%
%                                                                              %
%                                Disp first_param	                                           %
%                                                                              %
%******************************************************************************%

\chapter{Exercise \exercicenumber: disp\_first\_param}

\extitle{Display a parameter}
\exnumber{\exercicenumber}
\exscore{2}
\exfiles{disp\_first\_param.rb}
\exauthorize{All}

\makeheaderfiles

\begin{itemize}

\item Create a script \texttt{disp\_first\_param.rb} which, when executed, displays the first string passed as a parameter, followed by a newline. If there are no parameters, display \texttt{none} followed by a newline. 

\begin{42console}
	?> ./disp_first_param.rb | cat -e
	none$
	?> ./disp_first_param.rb "Beeblebrox" "Improbability" "Slartibartfast" | cat -e
	Beeblebrox$
	?>
\end{42console}

\end{itemize}

\hint{Google ARGV, array, condition if.}

%******************************************************************************%
%                                                                              %
%                                  Age                                         %
%                                                                              %
%******************************************************************************%

\chapter{Exercise \exercicenumber: age}

\extitle{Receive and modify a number}
\exnumber{\exercicenumber}
\exscore{2}
\exfiles{age.rb}
\exauthorize{All}

\makeheaderfiles

\begin{itemize}

\item Create a script \texttt{age.rb} that asks the user to enter their age, and then displays how old the user will be in 10 years, 20 years, and 30 years.

\begin{42console}
	?> ./age.rb
	Please tell me your age : 15
	You are currently 15 years old.
	In 10 years, you'll be 25 years old.
	In 20 years, you'll be 35 years old.
	In 30 years, you'll be 45 years old.
	?>
\end{42console}

\end{itemize}

\hint{Google string to\_i.}


%******************************************************************************%
%                                                                              %
%                                UPCASE_IT	                                           %
%                                                                              %
%******************************************************************************%

\chapter{Exercise \exercicenumber: UPCASE\_IT}

\extitle{Show in all caps}
\exnumber{\exercicenumber}
\exscore{2}
\exfiles{upcase\_it.rb}
\exauthorize{All}

\makeheaderfiles

\begin{itemize}

\item Create a script \texttt{upcase\_it.rb} which takes a character string as a parameter. When executed the script displays the string in all caps followed by a newline. If the number of parameters is different from 1, display \texttt{none} followed by a newline.

\begin{42console}
	?> ./upcase_it.rb | cat -e
	none$
	?> ./upcase_it.rb "don't panic" | cat -e
	DON'T PANIC$
	?> ./upcase_it.rb 'tHiS iS sO eAsY! - rUbY iS bAe' | cat -e
	THIS IS SO EASY! - RUBY IS BAE$
	?>
\end{42console}

\end{itemize}

\hint{Google upcase.}

%******************************************************************************%
%                                                                              %
%                                downcase_it	                                           %
%                                                                              %
%******************************************************************************%

\chapter{Exercise \exercicenumber: downcase\_it}

\extitle{Show in lowercase}
\exnumber{\exercicenumber}
\exscore{2}
\exfiles{downcase\_it.rb}
\exauthorize{All}

\makeheaderfiles

\begin{itemize}

\item Create a script \texttt{downcase\_it.rb} which takes a character string as a parameter. When executed, the script displays the string in lowercase followed by a newline. If the number of parameters is different from 1, display \texttt{none} followed by a newline.

\begin{42console}
	?> ./downcase_it.rb | cat -e
	none$
	?> ./downcase_it.rb "TRILLIAN" | cat -e
	trillian$
	?> ./downcase_it.rb 'tHiS iS sO eAsY! - rUbY iS bAe' | cat -e
	this is so easy! - ruby is bae$
	?>
\end{42console}

\end{itemize}

\hint{Try without asking for hints.}


%******************************************************************************%
%                                                                              %
%                                scan_it	                                           %
%                                                                              %
%******************************************************************************%

\chapter{Exercise \exercicenumber: scan\_it}

\extitle{Scan the text}
\exnumber{\exercicenumber}
\exscore{2}
\exfiles{scan\_it.rb}
\exauthorize{All}

\makeheaderfiles

\begin{itemize}

\item Create a script \texttt{scan\_it.rb} that takes two parameters. The first is a keyword to look for in a string. The second is the string to search. When executed, the program displays the number of occurrences of the keyword in the string. If the number of parameters is different from 2 or the first string does not appear in the second string, then display \texttt{none} followed by a newline. 

\begin{42console}
	?> ./scan_it.rb | cat -e
	none$
	?> ./scan_it.rb "the" | cat -e
	none$
	?> ./scan_it.rb "the" "these exercises in day01 are not the hardest ones we'll see \=P" | cat -e
	2$
	?>
\end{42console}

\end{itemize}

\hint{Try "string.scan(string)".}


%******************************************************************************%
%                                                                              %
%                               End of document                                %
%                                                                              %
%******************************************************************************%

\end{document}
