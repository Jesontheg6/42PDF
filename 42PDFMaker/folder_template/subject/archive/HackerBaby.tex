% vim: set ts=4 sw=4 tw=80 noexpandtab:
%******************************************************************************%
%                                                                              %
%                   HackerBaby.tex                                             %
%                   Made by: 42 staff                                          %
%                                                                              %
%******************************************************************************%

\documentclass{42-en}

%******************************************************************************%
%                                                                              %
%                                   Prologue                                   %
%                                                                              %
%******************************************************************************%

\begin{document}

\title{Hacker, Baby}
\subtitle{Show us your skillz}

\member {Kai}{kai@42.us.org}

\summary
{
This project is the doorway into the full pathway of HackHighSchool programming challenges. We are checking to see if you know a few things - how to repeat, how to make decisions, how to encapsulate code. Can you take data into your program; check the docs for options; discover built in data structures; and use logic to structure a program?
}

\maketitle

\tableofcontents

% %Initialisation des headers d'exercices
% \daypiscine{02}



%******************************************************************************%
%                                                                              %
%                               Introduction                                   %
%                                                                              %
%******************************************************************************%

\chapter{Introduction}

Welcome to HackHighSchool!

Take a deep breath, look around. This is your oasis for expressing creativity through code. Here is a place for hanging out with 
friends who have the skills to make things happen. Take a moment to caress the edges of your lovely iMac - aren't they wonderful? Take care of the shared lab space so that we can share it with many people to come;
that includes especially, most importantly, no drinks on the table. Keep only water in the lab and keep it on the
floor. Capiche.\\

The first checkpoint of HackHighSchool is to learn the basic controls of how to use a programming language. It's
like learning how to drive a bicycle or a car: there's a few features shared between all vehicles. Turn right,
turn left, move forward, slow down, stop.\\

We'd love for you to be bilingual or even multilingual programmers. Many of the projects we are going to offer are
open-language: you can turn in a solution in any programming language you like, so long as it outputs the right
program behavior. But our first stop is Ruby. We'll give everyone a shared experience by learning some Ruby syntax
to start it all off.\\


%******************************************************************************%
%                                                                              %
%                                 Vogsphere                                    %
%                                                                              %
%******************************************************************************%

\chapter{Vogsphere}

Quick refresher:
\begin{enumerate}

	\item From your project page on intra, copy the Git Repository link. Now, in the terminal type "git clone " and paste the link. After the link, write a name for the new folder. Cloning your Git repository always creates a new folder.
	\item cd into the folder you just created and from now on, save your work there. Use the command "mkdir <name>" to create new folders. Put each puzzle from this project in a folder with the same name.
	\item Each day, turn in your work so far by typing three commands in order: 
	\begin{itemize}
		\item git add *
		\item git commit -m "<your comments here>"
		\item git push
	\end{itemize}
	\warn{If you have an error during the git push, you may need to refresh your authentication ticket. Do this by typing "kinit <username>" and then typing your intra password.}

\end{enumerate}

This is how your directory structure should look:\\

\begin{42console}
	?> pwd
	~/kai/hackhighschool
	?> ls
	first_day
	hacker_baby
	information_libre
	?> cd hacker_baby & ls
	ex00
	ex01
	ex02
	ex03
	ex04
	ex05
	ex06
	?> cd ex00 & ls
	ex00.rb
\end{42console}

Inside the project folder are six folders names ex00 ... ex06. Inside each ex0X folder is a .rb or .py file with the same name.

%******************************************************************************%
%                                                                              %
%                                Guidelines                                    %
%                                                                              %
%******************************************************************************%

\chapter{Guidelines}

\begin{itemize}

  \item This project is corrected by Moulinette, our computer grader.
 
  \item Read the examples carefully. Your output should match the examples exactly as well as fulfill the instructions.

  \hint{You can retry the project as many times as you need until you get it right! You don't need to go through the peer correction process to see your result.}

  \warn{
  	Don't use any libraries or gems that you have to install yourself; they will not be available on the computer that grades you.
  }

\end{itemize}

\newpage


\startexercices


%******************************************************************************%
%                                                                              %
%                              What is your name?                              %
%                                                                              %
%******************************************************************************%

\chapter{Exercise \exercicenumber: What is your name}

\extitle{description}
\exnumber{\exercicenumber}
\exfiles{ex00.rb}

\makeheaderfiles


\begin{itemize}

\item Create a script \texttt{ex00.rb} which asks your name and greets you with it.

\begin{42console}
	?> ruby ex00.rb
	Hello hacker, what is your name?
	?> O'Brian
	Welcome, O'Brian.
\end{42console}

\end{itemize}


%******************************************************************************%
%                                                                              %
%                                Who Goes There                                %
%                                                                              %
%******************************************************************************%

\chapter{Exercise \exercicenumber: Who Goes There}

\extitle{description}
\exnumber{\exercicenumber}
\exfiles{ex01.rb}

\makeheaderfiles

\begin{itemize}

\item Create a script \texttt{ex01.rb} which asks your name and only greets you if your name is “Daenerys of the House Targaryen, the First of Her Name, The Unburnt, Queen of the Andals, the Rhoynar and the First Men, Queen of Meereen, Khaleesi of the Great Grass Sea, Protector of the Realm, Lady Regnant of the Seven Kingdoms, Breaker of Chains and Mother of Dragons” or "DHTFHNUQARFMQMKGSPRLRSKBCMD" for short. 
\item Otherwise, if your name is "Dany", the program replies "Dany who?". 
\item For any other name, the program replies "Move along, now."

\begin{42console}
	?> ruby ex01.rb
	Who goes there?
	?> DHTFHNUQARFMQMKGSPRLRSKBCMD
	Welcome, Daenerys.
\end{42console}

\begin{42console}
	?> ruby ex01.rb
	Who goes there?
	?> Dany
	Dany who?
\end{42console}

\begin{42console}
	?> ruby ex01.rb
	Who goes there?
	?> Jaqen H'gar
	Move along, now.
\end{42console}

\end{itemize}

%******************************************************************************%
%                                                                              %
%                                 ARR, Matey                                   %
%                                                                              %
%******************************************************************************%

\chapter{Exercise \exercicenumber: ARR Matey}

\extitle{description}
\exnumber{\exercicenumber}
\exfiles{ex02.rb}

\makeheaderfiles

\begin{itemize}

\item Create a script \texttt{ex02.rb} which takes a sentence worth of command-line arguments, splits them into an array, and then prints them each out on a different line along with the corresponding index of the array.
\item Next, sort the array by word length and reverse it, printing just the words in descending order of length.

\begin{42console}
	?> ruby ex02.rb ruby-doc.org shows comprehensive functions with arrays and strings :\)
	Argv of 0 is ruby-doc.org
	Argv of 1 is shows
	Argv of 2 is comprehensive
	Argv of 3 is functions
	Argv of 4 is with
	Argv of 5 is arrays
	Argv of 6 is and
	Argv of 7 is strings
	Argv of 8 is :)
	comprehensive
	ruby-doc.org
	functions
	strings
	arrays
	shows
	with
	and
	:)
	?>
\end{42console}

\end{itemize}

%******************************************************************************%
%                                                                              %
%                                Conditional Sum                               %
%                                                                              %
%******************************************************************************%

\chapter{Exercise \exercicenumber: Conditional Sum}

\extitle{description}
\exnumber{\exercicenumber}
\exfiles{ex03.rb}

\makeheaderfiles

\begin{itemize}

\item From Project Euler, a great resource for programming practice:\\
https://projecteuler.net/problem=1
\item If we list all the natural numbers below 10 that are multiples of 3 or 5, we get 3, 5, 6 and 9. The sum of these multiples is 23.
\item Create a script \texttt{ex03.rb} which finds the sum of all the multiples of 3 or 5 below the number given as a command line argument.
\item For this version, if given a negative number you must also find the sum of all multiples of 3 and 5 between that number and zero.

\begin{42console}
	?> ruby ex03.rb 42
	408
	?>
\end{42console}

\begin{42console}
	?> ruby ex03.rb 420
	40950
	?>
\end{42console}

\begin{42console}
	?> ruby ex03.rb 4242
	4198308
	?>
\end{42console}

\begin{42console}
	?> ruby ex03.rb -10
	-23
	?>
\end{42console}

\begin{42console}
	?> ruby ex03.rb 0
	0
	?>
\end{42console}

\end{itemize}


%******************************************************************************%
%                                                                              %
%                                Prime Suspects                                %
%                                                                              %
%******************************************************************************%

\chapter{Exercise \exercicenumber: Prime Suspects}

\extitle{description}
\exnumber{\exercicenumber}
\exfiles{ex04.rb}

\makeheaderfiles

\begin{itemize}

\item Create a script \texttt{ex04.rb} that prints the prime factors, in increasing order, of the number given as an argument.
\item If the input given is not a number or is less than one, print only a newline.

\begin{42console}
	?> ruby ex04.rb 29
	29
	?>
\end{42console}

\begin{42console}
	?> ruby ex04.rb 242
	2,11,11
	?>
\end{42console}

\begin{42console}
	?> ruby ex04.rb 60
	2,2,3,5
	?>
\end{42console}

\begin{42console}
	?> ruby ex04.rb nineteen ninety six
	?>
\end{42console}

\begin{42console}
	?> ruby ex04.rb 1
	1
	?>
\end{42console}

\end{itemize}

%******************************************************************************%
%                                                                              %
%                                    Memorize                                  %
%                                                                              %
%******************************************************************************%

\chapter{Exercise \exercicenumber: Memorize}

\extitle{description}
\exnumber{\exercicenumber}
\exfiles{ex05.rb}
\exnotes{Use the capitols.txt file provided on the project page. You do not need to turn that file in, but you can include it in your repository.}
\makeheaderfiles

\begin{itemize}

\item Create a script \texttt{ex05.rb} which reads in the provided comma-delimited file of US States and capitals and stores this information in a hashtable.
\item Next, on an infinite loop, print "Ready: " and wait for the user to enter the name of a state or capital. For each query print out the associated capital or state and go back to Ready state.
\item The program exits when the user types "Done". If the input is invalid, answer "nil".

\begin{42console}
	?> ruby ex05.rb capitals.txt
	Ready: Arizona
	Phoenix
	Ready: Montana
	Helena
	Ready: MacaroniAndCheese
	nil
	Ready: Pierre
	South Dakota
	Ready: Done
	?>
\end{42console}

\end{itemize}

%******************************************************************************%
%                                                                              %
%                             Do you even... 101010                            %
%                                                                              %
%******************************************************************************%

\chapter{Exercise \exercicenumber: Do You Even 101010}

\extitle{description}
\exnumber{\exercicenumber}
\exfiles{ex06.rb}

\makeheaderfiles

\begin{itemize}

\item Create a script \texttt{ex06.rb} which takes in a number in base 10 and prints out its equivalent in base 2 (binary).

\begin{42console}
	?> ruby ex06.rb 94555
	10111000101011011
	?>
\end{42console}

\hint {There are two types of people in the world: those who understand binary, and those who \texttt donut.}
\end{itemize}

%******************************************************************************%
%                                                                              %
%                               End of document                                %
%                                                                              %
%******************************************************************************%

\end{document}