 
% vim: set ts=4 sw=4 tw=80 noexpandtab:
%******************************************************************************%
%                                                                              %
%                   MySpace.tex                                                %
%                   Made by: 42 staff                                          %
%                                                                              %
%******************************************************************************%

\documentclass{42-en}

%******************************************************************************%
%                                                                              %
%                                   Prologue                                   %
%                                                                              %
%******************************************************************************%

\begin{document}

\title{A Room of One's Own:}
\subtitle{Exploration of HTML/CSS}

\member {Kai}{kai@42.us.org}

\summary
{
Build a webiste like it's 1999!
}

\maketitle

\tableofcontents

%Initialisation des headers d'exercices

%******************************************************************************%
%                                                                              %
%                                Preamble                                      %
%                                                                              %
%******************************************************************************%

\chapter{Preamble}

\textit{From a Wikipedia article related to the history of \texttt{hypertext} technology.}\\

"As We May Think" is a 1945 essay by Vannevar Bush which has been described as visionary and 
influential, anticipating many aspects of information society. It was first published in The 
Atlantic in July 1945 and republished in an abridged version in September 1945—before and after 
the atomic bombings of Hiroshima and Nagasaki. Bush expresses his concern for the direction of 
scientific efforts toward destruction, rather than understanding, and explicates a desire for a 
sort of collective memory machine with his concept of the memex that would make knowledge more 
accessible, believing that it would help fix these problems. Through this machine, Bush hoped to 
transform an information explosion into a knowledge explosion.\\

"As We May Think" predicted (to some extent) many kinds of technology invented after its 
publication, including hypertext, personal computers, the Internet, the World Wide Web, speech 
recognition, and online encyclopedias such as Wikipedia: "Wholly new forms of encyclopedias will 
appear, ready-made with a mesh of associative trails running through them, ready to be dropped 
into the memex and there amplified." Bush envisioned the ability to retrieve several articles or 
pictures on one screen, with the possibility of writing comments that could be stored and recalled
 together. He believed people would create links between related articles, thus mapping the 
 thought process and path of each user and saving it for others to experience. Wikipedia is one 
 example of how this vision has been realized, allowing users to link words to other related 
 topics, while browser user history maps the trails of the various possible paths of interaction.
  Bush's article also laid the foundation for new media. Doug Engelbart came across the essay 
  shortly after its publication, and keeping the memex in mind, he began work that would 
  eventually result in the invention of the mouse, the word processor, the hyperlink and concepts
   of new media for which these groundbreaking inventions were merely enabling technologies.\\

Today, storage has greatly surpassed the level imagined by Vannevar Bush,\\

\texttt{The Encyclopedia Britannica could be reduced to the volume of a matchbox. 
A library of a 
million volumes could be compressed into one end of a desk.}\\

On the other hand, it still uses methods of indexing of information which Bush described as 
artificial:\\

\texttt{When data of any sort are placed in storage, they are filed alphabetically or numerically, 
and information is found (when it is) by tracing it down from subclass to subclass. It can be in 
only one place, unless duplicates are used.}\\

This description resembles popular file systems of modern computer operating systems (FAT, NTFS, 
ext3 when used without hard links and symlinks, etc.), which do not easily enable associative 
indexing as imagined by Bush.\\

Bush concludes his essay by stating that:\\

\texttt{The applications of science have built man a well-supplied house, and are teaching him to live healthily therein. They have enabled him to throw masses of people against one another with cruel weapons. They may yet allow him truly to encompass the great record and to grow in the wisdom of race experience. He may perish in conflict before he learns to wield that record for his true good. Yet, in the application of science to the needs and desires of man, it would seem to be a singularly unfortunate stage at which to terminate the process, or to lose hope as to the outcome.} \\

\startexercices


%******************************************************************************%
%                                                                              %
%                                    Goals                                     %
%                                                                              %
%******************************************************************************%

\chapter{Goals}

Create the page from which you can announce your presence in the world and make a contribution to the sum of human knowledge.

%******************************************************************************%
%                                                                              %
%                            General Instructions                              %
%                                                                              %
%******************************************************************************%

\chapter{General Instructions}

Create a HTML webpage with CSS stylizing elements. We will host it for some time on our own server; and in the future, you can take the same template and host it somewhere else as your personal website.

%******************************************************************************%
%                                                                              %
%                                Mandatory Part                                %
%                                                                              %
%******************************************************************************%

\chapter{Mandatory Part}

\begin{itemize}

\item Your page must have an attention-grabbing title.
\item It must have a background color.
\item Add at least one image which takes you to another website when clicked.
\item It should have some text in a custom color.
\item It should have a custom-style border surrounding some part (or all) of the page.
\item Add a copywright at the bottom right, with the copyright symbol, your name, and the current year in italic and with a monospace font.

\end{itemize}

%******************************************************************************%
%                                                                              %
%                                  Bonus Part                                  %
%                                                                              %
%******************************************************************************%

\chapter{Bonus Part}

\begin{itemize}

\item Google "cool things you can do with CSS" and implement three of them.
\item Add a menu bar and multiple pages.
\item Add some original content, like your artwork or writing projects from elsewhere.
\item Sign up for a hosting website and host it yourself.
\item Add links to some of the most interesting things you can find around the Web.

\warn {Keep it appropriate, I'm warning you!}
\hint {Have fun \& show off :) :) :)}

\end{itemize}

%******************************************************************************%
%                                                                              %
%                               End of document                                %
%                                                                              %
%******************************************************************************%

\end{document}
