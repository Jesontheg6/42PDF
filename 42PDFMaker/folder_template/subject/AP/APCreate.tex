
% vim: set ts=4 sw=4 tw=80 noexpandtab:
%******************************************************************************%
%                                                                              %
%                   APCreate.tex                                               %
%                   Made by: 42 staff                                          %
%                                                                              %
%******************************************************************************%

\documentclass{42-en}

%******************************************************************************%
%                                                                              %
%                                   Prologue                                   %
%                                                                              %
%******************************************************************************%

\begin{document}

\title{AP Computer Science Principles - Create Task}
\subtitle{Build something cute and clever}

\member {Kai}{kai@42.us.org}

\summary
{
This PDF is a lesson plan for group work preparing you to succeed at the Create task of the APCSCP portfolio.
}

\maketitle

\tableofcontents

%******************************************************************************%
%                                                                              %
%                                Discussion                                    %
%                                                                              %
%******************************************************************************%

\chapter{Discussion}

Check out the handouts and preparation resources available in \href{https://drive.google.com/drive/folders/1lHIS-9qzAWCZyexNDr3rtBcgj4lM-jgo?usp=sharing}{this folder}.

\section{Form a pair}

Form a pair with one other pair to work with on this project. You may choose to collaborate with a partner on planning your project code, or just on answering the questions in this part.

\section{Understand what you will turn in}

You will need to turn in the following:
\begin{itemize}
	\item A PDF copy of the program code. You make do this by copying your code into Pages or Google Docs and saving to PDF from there.
	\item A screencast video demonstrating how your program runs.
	\item Written responses.
\end{itemize}

It is important to remember that although you can work with a partner, you must have specific parts of the code that you can claim as your own.

\section{Survival Guide}

Start with the Code.org \href{https://drive.google.com/open?id=1-U99DQmmtXMP89qT9NkTVA8szV2nWn_Y}{"Create Part - Surivival Guide"}. Take some time to read this document carefully.\\

Discuss with your partner and the class the meaning of Algorithm and Abstraction in the context of the task guidelines. 

\section{Grade practice examples}

Choose three or more of the examples from \href{https://drive.google.com/drive/folders/11iHSUH9LRAdehW57zJhNGrhJXZaDu2ji?usp=sharing}{this folder}.\\

Using the \href{https://drive.google.com/file/d/0B8tSiu_FFz0IVTNzOFhlMGdOQnM/view?usp=sharing}{2018 Scoring Rubric}, give the example project a score out of 8 points.\\

\section{Check your grading against the key}

Use the answer keys in \href{https://drive.google.com/drive/folders/1Bucwdax7POi-oylK6FhLSF2YxpW8ylQd?usp=sharing}{this folder} to check your understanding of the scoring guidelines against the answer key. Talk about the answers until you are sure that you understand why each example received the grade that it did.

\section{Brainstorm ideas and think about coding platforms}

Do you have an idea for a project you would like to create?\\

You may have complex ambitions for what you want to create, and that's awesome! But keep it simple for the purpose of this task. Write the minimum viable product (MVP) and you can always expand on it to make something with more features at a later date.\\

A lot of the examples that you will see online are written on platforms like MIT App Inventor or Snap. We have some resources for getting started with those \href{https://drive.google.com/drive/folders/1WshQqwtCvUzPrkrBS7i9B0RAtSuwKdP_?usp=sharing}{here} if you like. They are drag-and-drop coding systems where you assemble loops and if statements through a visual editor. It's not in Ruby or Python, but the coding logic is similar everywhere.\\

There's no need to switch platforms to those! The following variations on HackHighSchool coding tree projects can be completed in Ruby or Python and will fulfill the requirements of the College Board more than adequately.

\begin{itemize}
	\item A game which you play directly in the terminal, either a board game / card game simulation with ASCII output or a text-based adventure. This corresponds to the "Command-Line Games" branch of H2S. 
	\item A computer graphics animation using the Processing platform, which we teach in the branch "Graphics with Processing". Processing has an \href{http://hello.processing.org/}{Hour of Code} video which can help get you started. Right now it works on our computers in the Python or Javascript versions only; there is a Ruby extension which we are working to install by February or can help get it set up on your own computer.
	\item An interactive website using Sinatra (for the Ruby language) or Flask (for Python), if you are already familiar with writing HTML. (This is the project called "A Night at the Bar"). These tools can be used to make a simple game played by clicking buttons and inputting text, or a website that generates some kind of information for the user.
\end{itemize}


%******************************************************************************%
%                                                                              %
%                               Coding                                         %
%                                                                              %
%******************************************************************************%

\chapter{Coding}

You have January, February and March to work on this coding project as well as studying topics for the exam. Start gradually by playing around with the platform you want to use and brainstorming ideas of programs you'd like to make with it.\\

You should log 12 hours of in-class time working on this project. Tell your mentor when you are dedicating time to it.\\

If you need a quiet place to make a screencast, we can have let you sit in a quiet corner of the lab.\\

Refer to the scoring rubric and the example answers where students lost points when writing your essay questions.\\

\end{document}

% ** Tools

% http://www.freemake.com/blog/5-best-sites-to-make-animated-video-trouble-free/
% http://www.creativebloq.com/infographic/tools-2131971
