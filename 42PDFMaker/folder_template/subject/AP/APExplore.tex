
% vim: set ts=4 sw=4 tw=80 noexpandtab:
%******************************************************************************%
%                                                                              %
%                   APExplore.tex                                              %
%                   Made by: 42 staff                                          %
%                                                                              %
%******************************************************************************%

\documentclass{42-en}

%******************************************************************************%
%                                                                              %
%                                   Prologue                                   %
%                                                                              %
%******************************************************************************%

\begin{document}

\title{AP Computer Science Principles - Explore Task}
\subtitle{Know How it's Scored!}

\member {Kai}{kai@42.us.org}

\summary
{
This PDF is a lesson plan for group work preparing you to succeed at the Explore task of the APCSCP portfolio.
}

\maketitle

\tableofcontents

%******************************************************************************%
%                                                                              %
%                                Group Work                                    %
%                                                                              %
%******************************************************************************%

\chapter{Group Work}

\section{Form a group}

Form a group of 3-5 peers to work with on this project. You don't need to form a group on Intra, as you will not be turning in the group work. If there are printouts available, take printouts of the Explore Task resources and go away from the computers to complete the group work.

\section{Document Review}

Study the resources provided to you at \href{https://curriculum.code.org/csp/csp-ap/5}{Code.org: AP CSP Performance Task Lesson 5} and in our \href{https://drive.google.com/open?id=0B8tSiu_FFz0IdzJSWFU2VE1lR3M}{Google Drive folder}.\\

Make sure that you have a checklist of the steps required to complete the Explore Task correctly, or make one of your own.\\

Make sure that you have a grading rubric which shows how points are allocated for each section of the Explore task.\\

With your group, look for confusing terms in the instructions, checklist, or rubric and help each other define each one of them.

\begin{itemize}
	\item In this context, what is a "computational artifact"?
	\item In this context, what is a "computating innovation"?
	\item What are some examples of appropriate sources for this project and how will you cite them?
	\item What is "data" and what questions do you need to answer about data for this project?
	\item Can you think of some computing tools which would be useful for creating this project's computational artifact?
\end{itemize} 

\section{Brainstorm topics}

Within your group, brainstorm lots of potential topics for AP Explore projects. Give examples of your favorite computing innovations.\\

Name a benefit and a potential drawback of each one.\\
What kind of data does it use, and for what purpose?\\

\section{Grade the examples}

Choose two or more of the examples from \href{https://drive.google.com/open?id=0BwtvDcLkkxXgMEtURjQwUTFqa0k}{this folder}.\\

Using the \href{https://drive.google.com/open?id=0BwtvDcLkkxXgb0NzNE90cjlOOVU}{2018 Scoring Rubric}, give the example project a score out of 8 points.\\

\section{Check your grading against the key}

Use the answer keys in \href{https://drive.google.com/open?id=0BwtvDcLkkxXgbGpvS3BQQ2otRGs}{this folder} to check your understanding of the scoring guidelines against the answer key. Talk about the answers until you are sure that you understand why each example received the grade that it did.

\section{Discuss plagiarism}

Take a moment to discuss with your group your definition of the word "plagiarism". \\

There are many infographics and videos about innovative technology available on the internet. What do you need to do to make sure that your computational artifact is your own work?\\

What are some examples of intellectual dishonesty that could happen in the explore task that you would be concerned about if you saw other students doing them?\\


%******************************************************************************%
%                                                                              %
%                               On your Own                                    %
%                                                                              %
%******************************************************************************%

\chapter{On Your Own}

You have two writing exercises to complete:
\begin{itemize}
	\item written\_response\_practice.txt
	\item eval\_online\_content.txt
\end{itemize}

This part will be peer reviewed once and instructor-reviewed once. 

\section{Practice written responses}

Make sure you are subscribed to the "APCSP \- Explore" project and the "APCSCP \- Explore Practice" project inside of it.\\

Search on YouTube for an example of a Computational Artifact (video) that some student has produced for this class in the past. This is for practice ;) We're not going to turn their videos in as our own.\\

Open your Vogsphere repository and create a text file called written\_response\_practice.txt.\\

Using the \href{https://drive.google.com/open?id=0BwtvDcLkkxXgQUlVeDZHaEZySWM}{written response template}, complete the written part of the Explore task as if this video was the Artifact part of your submission. \\

Skip section 2b, where it prompts you to "Describe your development process".\\

Cite your example video at the top of your document using a link and the name of the channel it came from.\\

\section{Practice internet research}

Create another text file called eval\_online\_content.txt.\\

You are going to evaluate the credibility of a informational web page of your choice, using this exercise from Georgetown: \href{https://www.library.georgetown.edu/tutorials/research-guides/evaluating-internet-content}{Evaluating Internet Resources}.\\

Cite the web resource you are evaluating at the top of your text document.\\

Next, either paste the questionnaire into your text file and fill out an answer for every question; or, write a paragraph for every heading ("Author", "Purpose", "Objectivity", and so on).\\

\end{document}

% ** Tools

% http://www.freemake.com/blog/5-best-sites-to-make-animated-video-trouble-free/
% http://www.creativebloq.com/infographic/tools-2131971
