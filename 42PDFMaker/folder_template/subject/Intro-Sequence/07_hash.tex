% vim: set ts=4 sw=4 tw=80 noexpandtab:
%******************************************************************************%
%                                                                              %
%                   hash.tex                                                   %
%                   Made by: 42 staff                                          %
%                                                                              %
%******************************************************************************%

\documentclass{42-en}

%******************************************************************************%
%                                                                              %
%                                   Prologue                                   %
%                                                                              %
%******************************************************************************%

\begin{document}

\title{Hashes \textasciitilde/\textasciitilde Dictionaries}
\subtitle{Learn to Categorize}

\member {Kai}{kai@42.us.org}

\summary
{
	Keep a friend at your side, and study in the path of inspired ones who have given their knowledge out for free. 
}

\maketitle

\tableofcontents

%******************************************************************************%
%                                                                              %
%                               Before you Start                               %
%                                                                              %
%******************************************************************************%

\chapter{Before you Start!}

Create your project folder:
	\begin{enumerate}
		\item From your project page on intra, copy the git repository link. Now, in the terminal type "git clone " and paste the link. After the link and before pressing enter, write a name for the new folder. Cloning your git repository always creates a new folder.
		\item cd into the folder you just created and from now on, save your work there. Use the command "mkdir <name>" to create new folders. Put each puzzle from this project in a folder with the same name.
	\end{enumerate}


%******************************************************************************%
%                                                                              %
%                              Coding Guidelines                               %
%                                                                              %
%******************************************************************************%

\chapter{Format your Code}

Each 42 challenge you turn in must adhere to the following format:

\begin{42rbcode}
#!/usr/bin/env ruby

# This is what my program does
# By <userid>

def function_a
 #code
end

def function_b
 #code
end

def main(ARGV)
 #main method
 function_a
 function_b
end

main(ARGV)
\end{42rbcode}

\begin{itemize}
	\item Always begin with the "\#!/usr/bin/env ruby" statement. This tells your terminal to run the program using Ruby. In python, the first line is "\#!/usr/bin/env python".
	\item Always add a comment stating what this program is for, some hints to help others use or understand it, and your name or intra ID.
	\item Do not write any code outside of functions except for one line, at the end of your program, which calls the main() function.
	\item The (ARGV) parameter is not always needed. In Python it is sys.argv.
\end{itemize}

\hint{Reference your chosen intro to coding class to learn about functions/methods (The keyword "def" means "define function...").}

\startexercices


%******************************************************************************%
%                                                                              %
%                                    Memorize                                  %
%                                                                              %
%******************************************************************************%

\chapter{Exercise 00: Memorize}

\turnindir{ex00}
\exfiles{ex00.rb or ex00.py}
\exnotes{Use the capitols.txt file provided on the project page. You do not need to turn that file in, but you can include it in your repository. Ruby \href{https://alvinalexander.com/blog/post/ruby/how-process-line-text-file-ruby}{Text files in Ruby} Python \href{https://docs.python.org/3/tutorial/inputoutput.html}{Input/Output}}
\makeheaderfiles

\begin{itemize}

\item Create a script \texttt{ex00.rb} which reads in the provided comma-delimited file of US States and capitals and stores this information in a hashtable.
\item Next, on an infinite loop, print "Ready: " and wait for the user to enter the name of a state or capital. For each query print out the associated capital or state and go back to Ready state.
\item The program exits when the user types "Done". If the input is invalid, answer "nil".

\begin{42console}
	?> ruby ex00.rb capitals.txt
	Ready: Arizona
	Phoenix
	Ready: Montana
	Helena
	Ready: MacaroniAndCheese
	nil
	Ready: Pierre
	South Dakota
	Ready: Done
	?>
\end{42console}

\end{itemize}


%******************************************************************************%
%                                                                              %
%                                  Part VII                                    %
%                                                                              %
%******************************************************************************%

\chapter{Exercise 01: Hashes/Dicts}

\turnindir{ex01}
\exfiles{ex01.rb or ex01.py}
\exnotes{Use the names.txt file provided on the project page. Ruby \href{https://ruby-doc.org/core-2.4.2/Hash.html}{Hash} Python \href{https://docs.python.org/2/tutorial/datastructures.html\#dictionaries}{Dictionary}}
\makeheaderfiles

Using the attached file names.txt, store the information in a hash or dictionary where first names are associated with last names.\\

Use your hashtable to identify which first names are shared by more than one student, mentor or admin in h2s. Print out each first name that repeats in the set followed by an array of the last names associated with that first name.

\begin{42console}
	?> ruby ex01.rb
	# There are many ways to take the input in... you could use gets() and copy/paste the whole thing in the terminal here
	Elliot (2): [Tregoning, vanHeuman]
\end{42console}

* Note: This is not a real example. Eliot vanHeuman spells his name with one 'l' and Elliot Tregoning spells it with two. ;-)


%******************************************************************************%
%                                                                              %
%                               Grade your work                                %
%                                                                              %
%******************************************************************************%

\chapter{Grade your work!}

Turn in your work by typing three commands in order: 
\begin{itemize}
	\item git add *
	\item git commit -m "<your comments here>"
	\item git push
	\item If you have an error during the git push, you may need to refresh your authentication ticket. Do this by typing "kinit <username>" and then typing your intra password.
\end{itemize}

Then, go to your project page and click "Set the project as finished".
Next click "Subscribe to defense" and schedule two peer corrections.
If you run out of correction points (check your number on your profile page!), it means you need to open correction slots and correct other people in return. :)

\end{document}