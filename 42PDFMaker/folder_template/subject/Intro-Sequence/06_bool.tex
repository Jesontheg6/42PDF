% vim: set ts=4 sw=4 tw=80 noexpandtab:
%******************************************************************************%
%                                                                              %
%                   bool.tex                                                   %
%                   Made by: 42 staff                                          %
%                                                                              %
%******************************************************************************%

\documentclass{42-en}

%******************************************************************************%
%                                                                              %
%                                   Prologue                                   %
%                                                                              %
%******************************************************************************%

\begin{document}

\title{Boolean True-False}
\subtitle{Learn to Decide}

\member {Kai}{kai@42.us.org}

\summary
{
	Keep a friend at your side, and study in the path of inspired ones who have given their knowledge out for free. 
}

\maketitle

\tableofcontents

%******************************************************************************%
%                                                                              %
%                               Before you Start                               %
%                                                                              %
%******************************************************************************%

\chapter{Before you Start!}

Create your project folder:
	\begin{enumerate}
		\item From your project page on intra, copy the git repository link. Now, in the terminal type "git clone " and paste the link. After the link and before pressing enter, write a name for the new folder. Cloning your git repository always creates a new folder.
		\item cd into the folder you just created and from now on, save your work there. Use the command "mkdir <name>" to create new folders. Put each puzzle from this project in a folder with the same name.
	\end{enumerate}


%******************************************************************************%
%                                                                              %
%                              Coding Guidelines                               %
%                                                                              %
%******************************************************************************%

\chapter{Format your Code}

Each 42 challenge you turn in must adhere to the following format:

\begin{42rbcode}
#!/usr/bin/env ruby

# This is what my program does
# By <userid>

def function_a
 #code
end

def function_b
 #code
end

def main(ARGV)
 #main method
 function_a
 function_b
end

main(ARGV)
\end{42rbcode}

\begin{itemize}
	\item Always begin with the "\#!/usr/bin/env ruby" statement. This tells your terminal to run the program using Ruby. In python, the first line is "\#!/usr/bin/env python".
	\item Always add a comment stating what this program is for, some hints to help others use or understand it, and your name or intra ID.
	\item Do not write any code outside of functions except for one line, at the end of your program, which calls the main() function.
	\item The (ARGV) parameter is not always needed. In Python it is sys.argv.
\end{itemize}

\hint{Reference your chosen intro to coding class to learn about functions/methods (The keyword "def" means "define function...").}

\startexercices


%******************************************************************************%
%                                                                              %
%                                   Part VI                                    %
%                                                                              %
%******************************************************************************%

\chapter{Exercise 0: True, False, and nil/None}

\turnindir{ex00}
\exfiles{ex00.rb or ex00.py}
\exnotes{Ruby \href{https://ruby-doc.org/core-2.4.2/nilClass.html}{nilClass}, \href{https://ruby-doc.org/core-2.4.2/TrueClass.html}{TrueClass}, \href{https://ruby-doc.org/core-2.2.0/FalseClass.html}{FalseClass} Python \href{https://docs.python.org/2/library/constants.html}{Constants}, \href{https://docs.python.org/2/library/stdtypes.html}{Types}}
\makeheaderfiles

Using the three provided arrays:\\

Ruby\\
\noindent [false,true,true,nil,true,nil,nil,false,false,nil,true,false]\\
\noindent [or,or,or,==,!=,==,and,==,!=,and,!=,and]\\
\noindent [false,false,nil,nil,true,true,false,true,nil,false,true,nil]\\

Python\\
\noindent [False,True,True,None,True,None,None,False,False,None,True,False]\\
\noindent [or,or,or,==,!=,==,and,==,!=,and,!=,and]\\
\noindent [False,False,None,None,True,True,False,True,None,False,True,None]\\

Write a program which tests boolean logic by combining one element from each array into an equation and printing the full equation with its result. You can either use the arrays as given or have your program randomize them.

\begin{42console}
	?> ruby ex00.rb
	false or false => false
	true or false => true
	true or false => true
	nil == nil => true
	true != true => false
	...
\end{42console}

\hint{There are several ways to do this. Sophisticated ways include .send or getattr(). Simpler ways involve using a series of if statements to perform the operation requested. Practicing these equations in the terminal by typing "irb" or "python" may be fun.}

%******************************************************************************%
%                                                                              %
%                               Grade your work                                %
%                                                                              %
%******************************************************************************%

\chapter{Grade your work!}

Turn in your work by typing three commands in order: 
\begin{itemize}
	\item git add *
	\item git commit -m "<your comments here>"
	\item git push
	\item If you have an error during the git push, you may need to refresh your authentication ticket. Do this by typing "kinit <username>" and then typing your intra password.
\end{itemize}

Then, go to your project page and click "Set the project as finished".
Next click "Subscribe to defense" and schedule two peer corrections.
If you run out of correction points (check your number on your profile page!), it means you need to open correction slots and correct other people in return. :)

\end{document}