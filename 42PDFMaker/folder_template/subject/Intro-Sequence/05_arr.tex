% vim: set ts=4 sw=4 tw=80 noexpandtab:
%******************************************************************************%
%                                                                              %
%                   arr.tex                                                    %
%                   Made by: 42 staff                                          %
%                                                                              %
%******************************************************************************%

\documentclass{42-en}

%******************************************************************************%
%                                                                              %
%                                   Prologue                                   %
%                                                                              %
%******************************************************************************%

\begin{document}

\title{Arrays \textasciitilde/\textasciitilde Lists}
\subtitle{Learn to Sort}

\member {Kai}{kai@42.us.org}

\summary
{
	Keep a friend at your side, and study in the path of inspired ones who have given their knowledge out for free. 
}

\maketitle

\tableofcontents

%******************************************************************************%
%                                                                              %
%                               Before you Start                               %
%                                                                              %
%******************************************************************************%

\chapter{Before you Start!}

Create your project folder:
	\begin{enumerate}
		\item From your project page on intra, copy the git repository link. Now, in the terminal type "git clone " and paste the link. After the link and before pressing enter, write a name for the new folder. Cloning your git repository always creates a new folder.
		\item cd into the folder you just created and from now on, save your work there. Use the command "mkdir <name>" to create new folders. Put each puzzle from this project in a folder with the same name.
	\end{enumerate}


%******************************************************************************%
%                                                                              %
%                              Coding Guidelines                               %
%                                                                              %
%******************************************************************************%

\chapter{Format your Code}

Each 42 challenge you turn in must adhere to the following format:

\begin{42rbcode}
#!/usr/bin/env ruby

# This is what my program does
# By <userid>

def function_a
 #code
end

def function_b
 #code
end

def main(ARGV)
 #main method
 function_a
 function_b
end

main(ARGV)
\end{42rbcode}

\begin{itemize}
	\item Always begin with the "\#!/usr/bin/env ruby" statement. This tells your terminal to run the program using Ruby. In python, the first line is "\#!/usr/bin/env python".
	\item Always add a comment stating what this program is for, some hints to help others use or understand it, and your name or intra ID.
	\item Do not write any code outside of functions except for one line, at the end of your program, which calls the main() function.
	\item The (ARGV) parameter is not always needed. In Python it is sys.argv.
\end{itemize}

\hint{Reference your chosen intro to coding class to learn about functions/methods (The keyword "def" means "define function...").}

\startexercices


%******************************************************************************%
%                                                                              %
%                                 ARR, Matey                                   %
%                                                                              %
%******************************************************************************%

\chapter{Exercise 0: ARR Matey}

\turnindir{ex00}
\exfiles{ex00.rb or ex00.py}

\makeheaderfiles

\begin{itemize}

\item Create a script \texttt{ex00.rb} which takes a sentence worth of command-line arguments, splits them into an array, and then prints them each out on a different line along with the corresponding index of the array.
\item Next, sort the array by word length and reverse it, printing just the words in descending order of length.

\begin{42console}
	?> ruby ex01.rb ruby-doc.org shows comprehensive functions with arrays and strings :\)
	Argv of 0 is ruby-doc.org
	Argv of 1 is shows
	Argv of 2 is comprehensive
	Argv of 3 is functions
	Argv of 4 is with
	Argv of 5 is arrays
	Argv of 6 is and
	Argv of 7 is strings
	Argv of 8 is :)
	comprehensive
	ruby-doc.org
	functions
	strings
	arrays
	shows
	with
	and
	:)
	?>
\end{42console}

\end{itemize}


%******************************************************************************%
%                                                                              %
%                                    Part V                                    %
%                                                                              %
%******************************************************************************%

\chapter{Exercise 1: Arrays/Lists}

\turnindir{ex01}
\exfiles{ex01.rb or ex01.py}
\exnotes{Ruby \href{https://ruby-doc.org/core-2.4.1/Array.html}{Array}, Python \href{https://docs.python.org/2/library/stdtypes.html\#sequence-types-str-unicode-list-tuple-bytearray-buffer-xrange}{List}, \href{https://docs.python.org/2/tutorial/datastructures.html}{More on Lists}}
\makeheaderfiles

Take in a set of numbers as command line arguments. Store them as an array and print out the min, max, mean, median, mode and range of the set.

\begin{42console}
	?> ruby ex01.rb 142 6 13 36 54 4 9 78 78 102
	Min: 4
	Max: 142
	Mean: 52.2
	Median: 45
	Mode: 78
	Range: 138
\end{42console}


%******************************************************************************%
%                                                                              %
%                               Grade your work                                %
%                                                                              %
%******************************************************************************%

\chapter{Grade your work!}

Turn in your work by typing three commands in order: 
\begin{itemize}
	\item git add *
	\item git commit -m "<your comments here>"
	\item git push
	\item If you have an error during the git push, you may need to refresh your authentication ticket. Do this by typing "kinit <username>" and then typing your intra password.
\end{itemize}

Then, go to your project page and click "Set the project as finished".
Next click "Subscribe to defense" and schedule two peer corrections.
If you run out of correction points (check your number on your profile page!), it means you need to open correction slots and correct other people in return. :)

\end{document}