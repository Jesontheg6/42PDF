% vim: set ts=4 sw=4 tw=80 noexpandtab:
%******************************************************************************%
%                                                                              %
%                   io.tex                                                     %
%                   Made by: 42 staff                                          %
%                                                                              %
%******************************************************************************%

\documentclass{42-en}

%******************************************************************************%
%                                                                              %
%                                   Prologue                                   %
%                                                                              %
%******************************************************************************%

\begin{document}

\title{Input-Output}
\subtitle{Learn to talk}

\member {Kai}{kai@42.us.org}

\summary
{
	Keep a friend at your side, and study in the path of inspired ones who have given their knowledge out for free. 
}

\maketitle

\tableofcontents

%******************************************************************************%
%                                                                              %
%                               Before you Start                               %
%                                                                              %
%******************************************************************************%

\chapter{Before you Start!}

Create your project folder:
\begin{enumerate}
	\item From your project page on intra, copy the git repository link. Now, in the terminal type "git clone " and paste the link. After the link and before pressing enter, write a name for the new folder. Cloning your git repository always creates a new folder.
	\item cd into the folder you just created and from now on, save your work there. Use the command "mkdir <name>" to create new folders. Put each puzzle from this project in a folder with the same name.
\end{enumerate}


%******************************************************************************%
%                                                                              %
%                              Coding Guidelines                               %
%                                                                              %
%******************************************************************************%

\chapter{Format your Code}

Each 42 challenge you turn in must adhere to the following format:

\begin{42rbcode}
#!/usr/bin/env ruby

# This is what my program does
# By <userid>

def function_a
 #code
end

def function_b
 #code
end

def main(ARGV)
 #main method
 function_a
 function_b
end

main(ARGV)
\end{42rbcode}

\begin{itemize}
	\item Always begin with the "\#!/usr/bin/env ruby" statement. This tells your terminal to run the program using Ruby. In python, the first line is "\#!/usr/bin/env python".
	\item Always add a comment stating what this program is for, some hints to help others use or understand it, and your name or intra ID.
	\item Do not write any code outside of functions except for one line, at the end of your program, which calls the main() function.
	\item The (ARGV) parameter is not always needed. In Python it is sys.argv.
\end{itemize}

\hint{Reference your chosen intro to coding class to learn about functions/methods (The keyword "def" means "define function...").}

\startexercices


%******************************************************************************%
%                                                                              %
%                              What is your name?                              %
%                                                                              %
%******************************************************************************%

\chapter{Exercise 0: What is your name}

\turnindir{ex00}
\exfiles{ex00.rb or ex01.py}
\exnotes{Ruby \href{https://ruby-doc.org/core-2.4.2/IO.html}{IO}, \href{https://ruby-doc.org/core-2.4.2/ARGF.html}{ARGF} Python \href{https://docs.python.org/2/library/functions.html}{Built-in Functions}, \href{https://docs.python.org/2/library/sys.html}{System}}
\makeheaderfiles

\begin{itemize}

\item Create a script \texttt{ex00.rb} which asks your name and greets you with it.

\begin{42console}
	?> ruby ex00.rb
	Hello hacker, what is your name?
	?> O'Brian
	Welcome, O'Brian.
\end{42console}

\end{itemize}


%******************************************************************************%
%                                                                              %
%                                   Part I                                     %
%                                                                              %
%******************************************************************************%

\chapter{Exercise 1: Poetry}

\turnindir{ex01}
\exfiles{ex01.rb or ex01.py}
\exnotes{Ruby \href{https://ruby-doc.org/core-2.4.2/IO.html}{IO}, \href{https://ruby-doc.org/core-2.4.2/ARGF.html}{ARGF} Python \href{https://docs.python.org/2/library/functions.html}{Built-in Functions}, \href{https://docs.python.org/2/library/sys.html}{System}}
\makeheaderfiles

Create a short Mad-Libs puzzle. It will take a command-line argument which sets the title of the madlib. Then, your program prompts your corrector for the following:
\begin{itemize}
\item an adjective (for example, "fruity")
\item a business (for example, "orchard")
\item an animal (for example, "bat")
\item a noise (for example, "click")
\end{itemize}
Then, print a version of "Old MacDonald Had a Farm" using the user input instead of some of the traditional words.

Example of the program running:

\begin{42console}
	?> ruby 4200_io.rb "Ode to Joy"
	Please input an adjective: fruity
	Please input a business: orchard
	Please input an animal: bat
	Please input a noise: click

	ODE TO JOY
	fruity Macdonald had a orchard, E-I-E-I-O
	and on that orchard he had a bat, E-I-E-I-O
	with a click click here
	and a click click there,
	here a click, there a click,
	everywhere a click click,
	fruity Macdonald had a orchard, E-I-E-I-O! 
\end{42console}

\texttt{Bonuses!} Finish these bells and whistles to get extra credit.
\begin{itemize}
	\item match the example exactly.
	\item change 'a to 'an' depending if the word starts with a vowel.
	\item put the title in call caps.
	\item capitalize the start of each line including the adjective.
	\item put other input words in lowercase if they were capitalized.
	\item print an error message if the input is empty, longer than one word, or otherwise nonsensical.
\end{itemize}

%******************************************************************************%
%                                                                              %
%                               Grade your work                                %
%                                                                              %
%******************************************************************************%

\chapter{Grade your work!}

Turn in your work by typing three commands in order: 
\begin{itemize}
	\item git add *
	\item git commit -m "<your comments here>"
	\item git push
	\item If you have an error during the git push, you may need to refresh your authentication ticket. Do this by typing "kinit <username>" and then typing your intra password.
\end{itemize}

Then, go to your project page and click "Set the project as finished".
Next click "Subscribe to defense" and schedule two peer corrections.
If you run out of correction points (check your number on your profile page!), it means you need to open correction slots and correct other people in return. :)

\end{document}