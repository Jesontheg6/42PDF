% vim: set ts=4 sw=4 tw=80 noexpandtab:
%******************************************************************************%
%                                                                              %
%                   str.tex                                                    %
%                   Made by: 42 staff                                          %
%                                                                              %
%******************************************************************************%

\documentclass{42-en}

%******************************************************************************%
%                                                                              %
%                                   Prologue                                   %
%                                                                              %
%******************************************************************************%

\begin{document}

\title{String}
\subtitle{Learn to Write}

\member {Kai}{kai@42.us.org}

\summary
{
	Keep a friend at your side, and study in the path of inspired ones who have given their knowledge out for free. 
}

\maketitle

\tableofcontents

%******************************************************************************%
%                                                                              %
%                               Before you Start                               %
%                                                                              %
%******************************************************************************%

\chapter{Before you Start!}

Create your project folder:
	\begin{enumerate}
		\item From your project page on intra, copy the git repository link. Now, in the terminal type "git clone " and paste the link. After the link and before pressing enter, write a name for the new folder. Cloning your git repository always creates a new folder.
		\item cd into the folder you just created and from now on, save your work there. Use the command "mkdir <name>" to create new folders. Put each puzzle from this project in a folder with the same name.
	\end{enumerate}


%******************************************************************************%
%                                                                              %
%                              Coding Guidelines                               %
%                                                                              %
%******************************************************************************%

\chapter{Format your Code}

Each 42 challenge you turn in must adhere to the following format:

\begin{42rbcode}
#!/usr/bin/env ruby

# This is what my program does
# By <userid>

def function_a
 #code
end

def function_b
 #code
end

def main(ARGV)
 #main method
 function_a
 function_b
end

main(ARGV)
\end{42rbcode}

\begin{itemize}
	\item Always begin with the "\#!/usr/bin/env ruby" statement. This tells your terminal to run the program using Ruby. In python, the first line is "\#!/usr/bin/env python".
	\item Always add a comment stating what this program is for, some hints to help others use or understand it, and your name or intra ID.
	\item Do not write any code outside of functions except for one line, at the end of your program, which calls the main() function.
	\item The (ARGV) parameter is not always needed. In Python it is sys.argv.
\end{itemize}

\hint{Reference your chosen intro to coding class to learn about functions/methods (The keyword "def" means "define function...").}

\startexercices


%******************************************************************************%
%                                                                              %
%                                   Part III                                   %
%                                                                              %
%******************************************************************************%

\chapter{Exercise 0: Strings}

\turnindir{ex00}
\exfiles{ex00.rb or ex00.py}
\exnotes{Ruby \href{https://ruby-doc.org/core-2.4.0/String.html}{String}, \href{https://ruby-doc.org/core-2.4.2/Regexp.html}{Regular Expressions} Python \href{https://docs.python.org/2/library/string.html}{String}, \href{https://docs.python.org/2/library/re.html}{Regular Expressions}}
\makeheaderfiles

Write a progam which performs three tasks in a row on the phrase given as a command line argument.\\

First, it prints out the phrase with eVeRy OtHeR lEtTeR cApItAlIzEd.

Then, print out the same string with every capitalized vowel replaced by an asterisk.

Next, run a check\_parentheses function which determines whether every open paren "(" in the phrase is balanced with a close paren ")". Print out "Balanced? True" or "Balanced? False" accordingly. 

\begin{42console}
	?> ruby ex00.rb "(whats going on()?)"
	(WhAtS gOiNg On()?)
	(Wh*tS g*iNg *n()?)
	Balanced? True
\end{42console}

\begin{42console}
	?> ruby ex00.rb "()()()()(((a)b)b)b"
	()()()()(((A)b)B)b
	()()()()(((*)b)B)b
	Balanced? True
\end{42console}

\begin{42console}
	?> ruby ex00.rb "((this doesn't match)"
	((ThIs DoEsN't MaTcH)
	((Th*s Do*sN't MaTcH)
	Balanced? False
\end{42console}

\hint {Use gsub or re.sub. Study a little about regex. Look up what string interpolation means \& use it all the time.}


%******************************************************************************%
%                                                                              %
%                             Do you even... 101010                            %
%                                                                              %
%******************************************************************************%

\chapter{Exercise 1: Do You Even 101010}

\turnindir{ex01}
\exfiles{ex01.rb or ex01.py}

\makeheaderfiles

\begin{itemize}

\item Create a script \texttt{ex06.rb} which takes in a number in base 10 and prints out its equivalent in base 2 (binary).

\begin{42console}
	?> ruby ex01.rb 94555
	10111000101011011
	?>
\end{42console}

\hint {There are two types of people in the world: those who understand binary, and those who \texttt donut.}
\end{itemize}


%******************************************************************************%
%                                                                              %
%                               Grade your work                                %
%                                                                              %
%******************************************************************************%

\chapter{Grade your work!}

Turn in your work by typing three commands in order: 
\begin{itemize}
	\item git add *
	\item git commit -m "<your comments here>"
	\item git push
	\item If you have an error during the git push, you may need to refresh your authentication ticket. Do this by typing "kinit <username>" and then typing your intra password.
\end{itemize}

Then, go to your project page and click "Set the project as finished".
Next click "Subscribe to defense" and schedule two peer corrections.
If you run out of correction points (check your number on your profile page!), it means you need to open correction slots and correct other people in return. :)

\end{document}