% vim: set ts=4 sw=4 tw=80 noexpandtab:
%******************************************************************************%
%                                                                              %
%                   Parseltongue_00.tex                                        %
%                   Made by: 42 staff                                          %
%                                                                              %
%******************************************************************************%

\documentclass{42-en}

%******************************************************************************%
%                                                                              %
%                                   Prologue                                   %
%                                                                              %
%******************************************************************************%

\begin{document}

\title{Parseltongue Piscine - Day00}
\subtitle{Learn to 42}

\member {Kai}{kai@42.us.org}
\member {Ru}{ru@42.us.org}

\summary
{
The first day we are just going to practice some command line navigation, and write and run your first program. \texttt{One small program - one giant leap in a programmer's life.}
}

\maketitle

\tableofcontents

%Initialisation des headers d'exercices

%******************************************************************************%
%                                                                              %
%                                Preamble                                      %
%                                                                              %
%******************************************************************************%

\chapter{Preamble}

\section{What Question Will You Ask?}
\href{https://www.youtube.com/watch?v=x2rS-ha8DbE}{The Ultimate Question}\\

Why does Deep Thought state that it will be destined to build another supercomputer... even more powerful than itself?\\

If you don't get it, maybe check out the \href{https://www.youtube.com/watch?v=aboZctrHfK8}{remake}! It's cool too.

\newpage

\section{About 42}

42 is a community space for self-directed, peer-to-peer learning.

\begin{enumerate}

\item \texttt{Community:} The people we know in-person, who interact with us in daily and ordinary ways, are really important in ways that online friends can never fill. We provide a community space where coders can come together to work together with other independent, hardworking and clever people at no cost. I hope you will find some people you really like here. :)

\item \texttt{Self-directed:} you will meet some older mentors here, who are 42 cadets that have volunteered to help you learn. They are not really teachers though. In fact a lot of them have only studied a little bit of python and will be learning at your side. What they do know is how to use a project-based inquiry method to teach themselves a new programming language. :) We learn by reading official documentation, tutorials, blog posts, books, and Q\&A sites like StackOverFlow. When we get tired of staring at our own code, then we go help someone else with theirs.

\item \texttt{Peer-to-peer:} The more we share information and knowledge with each other, the stronger community we have to build useful applications. Rumors are encouraged! Fill the message board up with tips and programming jokes, and message people working on the same project if you have a question about it. At the end of every project, you will grade each others' work face to face. Be courteous and rigorous. You must ask them to explain their code, and then you must give them any suggestions you have to improve it, so that your friends will learn from you and you from them.

\end{enumerate}

\startexercices


%******************************************************************************%
%                                                                              %
%                                      Goals                                   %
%                                                                              %
%******************************************************************************%

\chapter{Goals for Today}

\begin{itemize}

	\item Start using the Slack for communications with mentors and other students.

	\item Start using intra.42.fr and explore the site.

	\item Get your programming environment set up with an installation of Python3, a text editor, and a terminal.

	\item Learn how to navigate the terminal, and become familiar with the text editors built into the terminal.

	\item Write your first Python program!

	\item Figure out how to turn in your work

	\item Set up your first grading session (aka a "correction") with another student.

\end{itemize}

%******************************************************************************%
%                                                                              %
%                                      Slack                                   %
%                                                                              %
%******************************************************************************%

\chapter{Slack}

Your welcome email contains information about how to join our group message board
on Slack. Go ahead and set that up if you haven't logged in yet!

%******************************************************************************%
%                                                                              %
%                                      Intra                                   %
%                                                                              %
%******************************************************************************%

\chapter{Intra}

\begin{itemize}

	\item Go to intra.42.fr. Sign in with the username and password that came to your email, the one you logged into the computer with.

	\item The first page is your profile page. A few things to notice:
	\begin{itemize}
		\item Your level meter, starting at 0.0\%. As you turn in projects and have them approved by your peers you will level up! If you reach a high level you will be eligible to take our month-long C programming curriculum in the summer.
		\item Your "Correction Points" in the top left. At 42, there is a correction point economy. You must spend a correction point when a peer grades your work and you earn a correction point when you grade someone else's.
		\item The "Evaluations" section. It is empty right now, but if you have an appointment to grade someone's project or vice versa that will appear here.
		\item The "Projects" section. You will find links to the projects you are currently working on right here.
	\end{itemize}

	\item In the Evaluations section, click on the "Manage Slots" button. It will show you a calendar of the current week. Find the current date and time. Then, click and drag to open an availability slot for the last hour that you plan to be here. (If you will be here until 4, click and drag from 3-4 pm). You should open a slot like this each day that you come to 42.

	\item Now, notice grey icons on the left side of the page.
	\begin{itemize}
		\item Profile: The top, 42, and the head-and-shoulders icon will both take you to your profile home page.
		\item Projects: The graph icon. Click here and see the views in "All Projects" and "List projects". In the "All projects" map, you will start at the bottom with the First Day project, progress through a series of learn-to-code challenges, and then pick any branch of the tree that you would like to explore. If you already know how to code, you can use "List projects" to see which projects are recommended. You can register to any of these instead of the intro sequence.
		\item E-Learning: The movie strip icon links to a few videos and PDFs that are produced by 42 and available for reference. There are many more in the C curriculum (invisible to you), but most of them are not yet translated from French.
		\item Forum: The speech bubble icon links to a forum. It is currently mostly used by French students and not the Americans, but you are welcome to create your own special land there.
		\item The last three, Companies, Meta, and Shop will not be very relevant for you at this time.
	\end{itemize}

\end{itemize}

%******************************************************************************%
%                                                                              %
%                                   Coding Tools                               %
%                                                                              %
%******************************************************************************%

\chapter{Coding Tools}

Set up your programming environment by picking a program for each of three essential functions. Our favorites are listed below. If you want to install one which is not already on your computer, simply download the program and then drag its icon to your desktop or to a folder of your choice (just not Applications).

\begin{enumerate}
	\item Web browser
	\begin{itemize}
		\item Built-in: Safari
		\item Free: Firefox
		\item Free: Google Chrome
	\end{itemize}
	\item Terminal
	\begin{itemize}
		\item Built-in: Terminal
		\item Free: iTerm2
	\end{itemize}
	\item Text Editor
	\begin{itemize}
		\item Built-in in the terminal: Emacs or Vim
		\item Built-in: Xcode
		\item Free: Atom
		\item Free: Sublime
		\item Free: Visual Studio Code
	\end{itemize}
	\newpage
	\item Python3!

		From the project page, download the file called "setup.sh".
		Then, open your Terminal and type the following command:

		\begin{verbatim}
			chmod 744 ~/Downloads/setup.sh && sh ~/Downloads/setup.sh
		\end{verbatim}
		(and press enter).
		Let this script run for a long time. It will install Python3 and set that as the default on your computer. It also installs some tools which allow you to switch Python versions easily and install Python libraries when you need to.

	\item A Trusty Reference Book
	\begin{itemize}
		\item For Python 3, I recommend that you bookmark the freely distributed book \href{https://python.swaroopch.com/}{A Byte of Python} by Swaroop to use as your guide. It will help to explain many things to you along the way!
	\end{itemize}
\end{enumerate}


%******************************************************************************%
%                                                                              %
%                                    Terminal                                  %
%                                                                              %
%******************************************************************************%

\chapter{Terminal}

Let's start learning how to navigate through our terminal with command lines! Follow these steps and remember it because you'll need it for today's assignments! \\

First things first: Press "command + space" to bring up a search bar. Type "iterm2" and press "enter" to open the terminal. This is the window we use to communicate directly with the brains of the machine.
\begin{description}\itemsep3pt
		\item [pwd:] "Present Working Directory" For all the "Where am I?" moments. When you first open a terminal, you start in your Home directory, which is labelled with your username.
\begin{42console}
$ pwd\end{42console}
		\item [cd:] aka "Change Directory". "cd .." means go back a directory. "cd" by itself means go back to Home directory. Below we are moving into a directory called Desktop.
\begin{42console}
$ cd Desktop/\end{42console}
		\item [mkdir:] Short for "Make Directory". Creates a new empty directory.
\begin{42console}
$ mkdir ex00\end{42console}
		\item [ls or ls -a:] "list", list all the folders and files inside the current directory you are in.
\begin{42console}
$ ls\end{42console}
		\item [touch:] create files (files will usually end with the language you are using for example ruby is 42.rb, python is 42.py, text files are 42.txt)
\begin{42console}
$ cd ex00
$ touch 42.txt\end{42console}
		\item [vim:] to use an in terminal text editor to edit your files. Once inside vim, press "i" to edit and press ":wq" to save and exit the file. Try writing "Hello World" on the top and then exiting.
\begin{42console}
$ vim 42.txt\end{42console}
		\item [cat:] to preview the contents of the file in Terminal. It should show "Hello World".
\begin{42console}
$ cat 42.txt\end{42console}
		\item [cp:] "copy", copies files or directories. Here, we copy the file 42.txt and place it in the HackHighSchool/ directory. Don't forget to make the HackHighSchool directory first.
\begin{42console}
$ cp 42.txt HackHighSchool/\end{42console}
		\item [mv:] "move",  move a file into a directory, use mv with the source file as the first argument and the destination directory as the second argument. Here we move 42.txt into HackHighSchool/
\begin{42console}
$ mv 42.txt HackHighSchool/\end{42console}
		\item [rm or rm -rf:] "rm" to delete files and "rm -rf" to delete directories.
\begin{42console}
$ rm -rf HackHighSchool\end{42console}
		\item [ascii banner:] create an ascii art banner on your terminal
\begin{42console}
$ banner -w 35 COOL\end{42console}
	  \item [say:] Make your mac say whatever you want it to say.
\begin{42console}
$ say "HackHighSchool is super fun"\end{42console}
  \end{description}
	\warn{Be careful with rm and rm -rf, you might accidentally delete something you need!}
	\hint{use your tab button to autofill your commands! Trust me, it's super useful once you get the hang of it.}

%******************************************************************************%
%                                                                              %
%                                   Vogsphere                                  %
%                                                                              %
%******************************************************************************%

\chapter{Vogsphere}

Recommended reference \& optional effort: Michael Hartl's \href{https://www.learnenough.com/git-tutorial}{"Learn Enough Git to be Dangerous"}.

\begin{enumerate}

\item From your page for this project on intra, copy the "Git Repository" link. (The one that looks like vogsphere@vgs.42.us.org:intra/2017...). If you do not have a link yet, make sure you are registered to the project, and wait about 5 minutes while refreshing the page.

\item Now, in the terminal type "git clone <copied link> (space) <newfoldername>". Replace <link> with your pasted link and <newfoldername> with first\_day, as a name for the project folder.

\item cd into the newly created folder. Everything inside here can be uploaded to Vogsphere.

\item Complete the project requirements (see next page). All files for the project should go in the folder we just created.

\end{enumerate}

\hint{If you have an error during the git clone type "kinit <username>" and press enter. Then, type your intra password.}

%******************************************************************************%
%                                                                              %
%                                     Warmup                                   %
%                                                                              %
%******************************************************************************%

\chapter{Warmup before coding}

\turnindir{ex00}
\exfiles{42.txt}
\makeheaderfiles

Here is the assignment for today - it is very simple. Look for a beginner's guide on command line navigation to help you find out how to do these things. Recommended: Michael Hartl's \href{https://www.learnenough.com/command-line-tutorial}{"Learn Enough Command Line to be Dangerous"}.

\begin{itemize}

	\item Create an empty directory called "day00" on your Desktop.
	\item Create an empty text file named 42.txt on your Desktop.
	\item Create a hidden text file (42.txt) in the day00 directory.
	\item Inside the hidden text file, write down a phrase that is your favorite kind of compliment you like to receive.
	\item Go back to your desktop and edit your text file named 42.txt.
	\item Inside your text file, write the terminal commands you learned and what they do.
	\item On a new line, write the location of the file you are currently editing.
	\item Move/copy the 42.txt file into the day00 directory.
\end{itemize}


\subsection{Bonus!}
\begin{itemize}

	\item make the command line "sl" work on your computer.

\end{itemize}

\hint{use Google homebrew to learn how to install sl}

\startexercices
%******************************************************************************%
%                                                                              %
%                                        Hello 42                              %
%                                                                              %
%******************************************************************************%

\chapter{Exercise 0: The Ultimate Question}

\extitle{The Ultimate Question}
\exfiles{00\_deepthought.rb or 00\_deepthought.py}
\exnotes{\href{https://bit.ly/2HdBjyu}{Stack Overflow can often help}}
\makeheaderfiles

\begin{itemize}

\item Create a script \texttt{00\_deepthought.py} which prints out 'The answer to the Ultimate Question, of Life, the Universe and Everything, is "42"'.

\begin{42console}
	?> python 00_deepthought.py
	The answer to the Ultimate Question, of Life, the Universe and Everything, is "42".
\end{42console}

\end{itemize}

%******************************************************************************%
%                                                                              %
%                               Turning In Your Code                           %
%                                                                              %
%******************************************************************************%

\chapter{Turning in your code}

Open the Vogsphere PDF and reference the instructions there to turn in your work.

%******************************************************************************%
%                                                                              %
%                                 Corrections                                  %
%                                                                              %
%******************************************************************************%

\chapter{Corrections}

We will schedule corrections after each project, which means you will randomly paired with another student to review their work. It's not too difficult - just remember that the point of the correction is to look at and ask them about their actual code!

% \begin{itemize}
% 	\item On your project page, click the button "Subscribe to Defense." Choose a time slot from the screen that shows next.

% 	\item When the appointment time comes, look up your assigned corrector on Slack and send them a message. Find each other.

% 	\item Corrector, sit at the correctee's station and open a new web browser in Incognito mode. Log into your intra.

% 	\item Access the corrections page. Clone their Git repository into a new folder.

% 	\item Click "Begin correction" and follow the instructions on the correction page.

% 	\item Use this time to chat about the project and the rest of your life.

% 	\item Once the correction is finished, your corrector should remember to log out of Intra.

% 	\item After both the correction is complete, in order to finalize your score you must then provide feedback for the corrector on your project page. Go to your project page and click the "feedback" button for each completed correction.

% \end{itemize}

%******************************************************************************%
%                                                                              %
%                                 What's next?                                 %
%                                                                              %
%******************************************************************************%

\chapter{What's next?}

If you have extra time, here are a few things to do:

\begin{enumerate}
	\item Talk to your mentor about your previous coding experience and ask them for a challenging project to start with if you already know the basics of how to code in Ruby or Python, or if you want to work in a different programming language.

	\item Learn more about the command line, and Emacs or Vim.

	\item Learn more about the Git protocol and set up your own Github or Bitbucket account to save your code for yourself online.

	\item As always, you can move on to starting the next project right away. Beginners go to \href{https://projects.intra.42.fr/projects/h2s-intro-sequence}{H2S Intro Sequence.}

\end{enumerate}

%******************************************************************************%
%                                                                              %
%                               End of document                                %
%                                                                              %
%******************************************************************************%

\end{document}
