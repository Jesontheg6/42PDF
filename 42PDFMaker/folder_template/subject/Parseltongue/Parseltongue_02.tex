
% vim: set ts=4 sw=4 tw=80 noexpandtab:
%******************************************************************************%
%                                                                              %
%                   Parseltongue_02.tex                                        %
%                   Made by: 42 staff                                          %
%                                                                              %
%******************************************************************************%

\documentclass{42-en}

%******************************************************************************%
%                                                                              %
%                                   Prologue                                   %
%                                                                              %
%******************************************************************************%

\begin{document}

\title{Parseltongue Piscine - Day02}
\subtitle{Arrays; Numbers; Booleans}

\member {Kai}{kai@42.us.org}

\summary
{
Learn about arrays, indexing, for loops, ARGV, floats, integers, booleans, numeric bases, and irb
}

\maketitle

\tableofcontents

%Initialisation des headers d'exercices

\newpage

\bigskip

\centerline{\includegraphics[width=150mm]{images/dont_panic.jpeg}}

\centerline{\texttt{Eat, Sleep, Code, Repeat.}}

%******************************************************************************%
%                                                                              %
%                                 Don't Panic                                  %
%                                                                              %
%******************************************************************************%

\chapter{Don't Panic!}

Confused about how to begin? Not sure what the PDF means?\\

Don't worry, we are not perfect :)\\

And the PDFs are challenges - not walkthroughs. Team up with your group, your partner,
your mentor and your other peers to decipher what to do.\\

You are the master of your own destiny! Go forward and code the world !

\startexercices


%******************************************************************************%
%                                                                              %
%                                 Brainstorm                                   %
%                                                                              %
%******************************************************************************%

\chapter{Exercise 0: Brainstorm}
\extopics{ASCII Control Codes, Lists}
\exfiles{00\_brainstorm.rb or 00\_brainstorm.py}
\extitle{Brainstorm}
\exnotes{\href{http://effbot.org/zone/python-list.htm}{Lists}, \href{https://www.youtube.com/watch?v=H3s4129glHw}{Append List Method}}
\makeheaderfiles

Create a program called \texttt{00\_brainstorm.py} which imitates a version of the game Scattegories. Here are some goals for it:

\begin{itemize}
	\item Your game should have a built in list or tuple which contains a selection of categories. When the game begins, choose one of the categories from the list randomly. (Research: How to choose a random number between 0 and n?)
	\item Once the random category is chosen, the program displays it and loops 10 times to collect input from the user.
	\item Every answer the user gives is added to another list, the answers lists.
	\item After 10 entries, display each answer that was given.
	\item The answers should be displayed neatly. Print them in a column that is centered in the middle of the terminal, with a blank line between each answer.
	\item Bonus: Time how long it takes the user to enter 10 items for that category, and display the elapsed time at the end.
	\item Bonus: Draw a nice box around the answer table output.
\end{itemize}

\nextexercice
\newpage

%******************************************************************************%
%                                                                              %
%                                 ARR, Matey                                   %
%                                                                              %
%******************************************************************************%

\chapter{Exercise 1: ARR Matey}
\extopics{ARGV}
\exfiles{01\_arrmatey.rb or 01\_arrmatey.py}
\extitle{ARR, Matey}
\exnotes{\href{https://www.youtube.com/watch?v=FjxmUkNns-8}{Increment and Decrement}, \href{http://thomas-cokelaer.info/tutorials/python/data_structures.html}{Data Structures}}
\makeheaderfiles

\begin{itemize}

\item Create a script \texttt{01\_arrmatey.py} which takes a sentence worth of command-line arguments, splits them into an array, and then prints them each out on a different line along with the corresponding index of the array.
\item Next, sort the array by word length and reverse it, printing just the words in descending order of length.

\begin{42console}
	?> python 01_arrmatey.py docs.python.org/3/ has official info regarding every module :\)
	Argv of 0 is 01_arrmatey.py
	Argv of 1 is docs.python.org/3/
	Argv of 2 is has
	Argv of 3 is official
	Argv of 4 is info
	Argv of 5 is regarding
	Argv of 6 is every
	Argv of 7 is module
	Argv of 8 is :)
	docs.python.org/3/
	01_arrmatey.py
	regarding
	official
	module
	every
	info
	has
	:)
	?>
\end{42console}

\end{itemize}

\nextexercice
\newpage

%******************************************************************************%
%                                                                              %
%                                 Numeric Types                                %
%                                                                              %
%******************************************************************************%

\chapter{Exercise 2: Numeric Types}
\extopics{Variable types, floats}
\exfiles{02\_numtypes.rb or 02\_numtypes.py}
\exnotes{\href{https://docs.python.org/2/library/stdtypes.html}{Numeric Types} \href{https://www.tutorialspoint.com/python3/python_command_line_arguments.htm
}{Command Line Arguments}, \href{https://docs.python.org/3/library/functions.html}{Built-in Functions}}
\makeheaderfiles

Write a program that takes two numbers as command line arguments.\\

They will come into your program as Strings. Find a way to turn them into numbers with which you can perform math.\\

Divide the first number by the second one and print out both the integer quotient and the remainder.\\

Then, your program should declare and initialize four variables of different numeric types.
Print them out and use the built-in functions type() or .class to print the variable type of each.

\begin{42console}
	?> python 02\_numtypes.py 142 6
	142 divided by 6 equals 23 remainder 4
	Variable a contains : 10  which is of type: Integer
	Variable b contains: 56.99  which is of type: Float
	Variable c contains: 2+3i  which is of type: Complex
	...
\end{42console}


\nextexercice
\newpage


% %******************************************************************************%
% %                                                                              %
% %                                   Prime Suspects                             %
% %                                                                              %
% %******************************************************************************%

% \chapter{Exercise 3: Prime Suspects}

% \extitle{Prime Suspects}
% \exfiles{01\_primes.rb or 01\_primes.py}

% \makeheaderfiles

% \begin{itemize}

% \item Create a script \texttt{01\_primes.py} that prints the prime factors, in increasing order, of the number given as an argument.
% \item If the input given is not a number or is less than one, print only a newline.

% \begin{42console}
% 	?> python 01_primes.py 29
% 	29
% 	?>
% \end{42console}

% \begin{42console}
% 	?> python 01_primes.py 242
% 	2,11,11
% 	?>
% \end{42console}

% \begin{42console}
% 	?> python 01_primes.py 60
% 	2,2,3,5
% 	?>
% \end{42console}

% \begin{42console}
% 	?> python 01_primes.py nineteen ninety six
% 	?>
% \end{42console}

% \begin{42console}
% 	?> python 01_primes.py
% 	?>
% \end{42console}

% \end{itemize}


% %******************************************************************************%
% %                                                                              %
% %                              Project Euler                                   %
% %                                                                              %
% %******************************************************************************%

% \chapter{Exercise 2: Project Euler}

% \extitle{Project Euler}
% \exfiles{02\_euler.rb or 02\_euler.py}

% \makeheaderfiles

% \begin{itemize}

% \item From Project Euler, a great resource for programming practice:\\
% https://projecteuler.net/problem=1
% \item If we list all the natural numbers below 10 that are multiples of 3 or 5, we get 3, 5, 6 and 9. The sum of these multiples is 23.
% \item Create a script \texttt{02\_euler.py} which finds the sum of all the multiples of 3 or 5 below the number given as a command line argument.
% \item For this version, if given a negative number you must also find the sum of all multiples of 3 and 5 between that number and zero.

% \begin{42console}
% 	?> python 02_euler.py 42
% 	408
% 	?>
% \end{42console}

% \begin{42console}
% 	?> python 02_euler.py 420
% 	40950
% 	?>
% \end{42console}

% \begin{42console}
% 	?> python 02_euler.py 4242
% 	4198308
% 	?>
% \end{42console}

% \begin{42console}
% 	?> python 02_euler.py -10
% 	-23
% 	?>
% \end{42console}

% \begin{42console}
% 	?> python 02_euler.py 0
% 	0
% 	?>
% \end{42console}

% \end{itemize}

%******************************************************************************%
%                                                                              %
%                                   101010                                     %
%                                                                              %
%******************************************************************************%

\chapter{Exercise 3: 101010}
\extopics{Base systems (Binary, Hexadecimal, Octal)}
\extitle{101010}
\exfiles{03\_binary.rb or 03\_binary.py}
\exnotes{\href{https://www.youtube.com/watch?v=ZL-LhaaMTTE}{Binary and Hexidecimal} \href{https://www.youtube.com/watch?v=aW3qCcH6Dao}{Conversions}}
\makeheaderfiles

\begin{itemize}

\item Create a script \texttt{03\_binary.py} which takes in a number in base 10 and prints out its equivalent in base 2 (binary), in base 8 (octal), and in base 16 (hexadecimal).

\begin{42console}
	?> python 03_binary.py 94555
	10111000101011011
	270533
	1715B
	?>
\end{42console}

\hint {There are two types of people in the world: those who understand binary, and those who \texttt donut.}
\hint {Real hint: there is a built in function for this... search around!}
\end{itemize}

%******************************************************************************%
%                                                                              %
%                                      Bool                                    %
%                                                                              %
%******************************************************************************%

\chapter{Exercise 4: George Bool}
\extopics{Booleans, iPython}
\extitle{George Bool}
\exfiles{04\_bool.rb or 04\_bool.py}
\exnotes{\href{https://docs.python.org/2/library/constants.html}{Constants}, \href{https://docs.python.org/2/library/stdtypes.html}{Types}, \href{https://www.youtube.com/watch?v=9OK32jb_TdI}{Booleans}, \href{https://docs.python.org/3/tutorial/interpreter.html}{Python Interpreter}}
\makeheaderfiles

Using the three provided arrays:\\

Ruby\\
\noindent [false,true,true,nil,true,nil,nil,false,false,nil,true,false]\\
\noindent ["or","or","or","==","!=","==","and","==","!=","and","!=","and"]\\
\noindent [false,false,nil,nil,true,true,false,true,nil,false,true,nil]\\

Python\\
\noindent [False,True,True,None,True,None,None,False,False,None,True,False]\\
\noindent ["or","or","or","==","!=","==","and","==","!=","and","!=","and"]\\
\noindent [False,False,None,None,True,True,False,True,None,False,True,None]\\

Write a program which tests boolean logic by combining one element from each array into an equation and printing the full equation with its result. You can either use the arrays as given or have your program randomize them.

\begin{42console}
	?> ruby 04_bool.rb
	false or false => false
	true or false => true
	true or false => true
	nil == nil => true
	true != true => false
	...
\end{42console}

\hint{There are several ways to do this. Sophisticated ways include .send or getattr(). Simpler ways involve using a series of if statements to perform the operation requested. Practicing these equations in the terminal by typing "irb" or "python" would be a helpful way to start.}


\end{document}
